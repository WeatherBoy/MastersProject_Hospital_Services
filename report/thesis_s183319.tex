\documentclass{article}
\usepackage[english]{babel}
\usepackage[top=2cm, bottom=3cm, left=2cm, right=2cm]{geometry}

% ## Inserting Code ##
\usepackage{pythonhighlight} % <-- for inserting Python code

% ## Nice referencing ##
\usepackage{csquotes}   % <-- for biblatex
\usepackage{biblatex}
\addbibresource{Bibliography.bib}

% ## Math ##
\usepackage{amssymb}
\usepackage{amsmath}
\usepackage{amsthm}
\usepackage{cancel} % <-- the big slash over equations
\usepackage{breqn}  % <-- Automatic wrapping of long equations (good stuff)

% ## For the graphical aspect ##
\usepackage{float}  % <-- superior package for figure handling
\usepackage[dvipsnames]{xcolor}
\usepackage{graphicx} % <-- Required for inserting images
\usepackage{wrapfig} % <-- In case you want to wrap text around figures (can look quite pro)
\usepackage{arydshln}   % <-- For horizontal, vertical, etc. lines (for tables and matrices)
\usepackage{subcaption} % <-- For subfigures
\usepackage[many]{tcolorbox} % <-- For making boxes around text

% ## Referencing ##
\usepackage{fancyref}   % <-- Supposed to make referencing better, don't know where it's used
\usepackage{hyperref}   % <-- has \autoref{} and all kinds of other goodies
\usepackage{url}   % <-- web addresses are handled properly when URL referencing

% ## Utility ##
\usepackage{enumitem}   % <-- enumerate, itemize and description
\usepackage{pdfpages}   % <-- for including external PDFs
\usepackage{titlesec}   % <-- (used for the header) section and chapter titles manipulation
\usepackage{subfiles}   % <-- smart for BIG projects where you include multiple subfiles
\usepackage{comment}    % <-- multi-line comments
\usepackage{fontawesome5}   % <-- add cool logos to your report
\usepackage{sectsty}    % <-- can help with generalising section design

\usepackage{fancyhdr}   % <-- for the header and footer
\pagestyle{fancy}   % <-- used to great extent for the title page header

% Make github logo
\usepackage{fontawesome5}
\newcommand{\github}[1]{%
   \href{#1}{\faGithubSquare}%
}

% For math definitions
\newtheorem{definition}{Definition}[section]
\newtheorem{theorem}{Theorem}[section]
\newtheorem{corollary}{Corollary}[section]
\tcolorboxenvironment{definition}{
  sharp corners,
  boxrule=0.4pt,
  colback=white,
  before skip=\topsep,
  after skip=\topsep,
}
\tcolorboxenvironment{theorem}{
  sharp corners,
  boxrule=0.4pt,
  colback=white,
  before skip=\topsep,
  after skip=\topsep,
}
\tcolorboxenvironment{corollary}{
  sharp corners,
  boxrule=0.4pt,
  colback=white,
  before skip=\topsep,
  after skip=\topsep,
}


% ## For pseudocode ##
\usepackage{listings}
\lstset{
  basicstyle=\small\ttfamily,
  numbers=left,
  numberstyle=\tiny,
  stepnumber=1,
  frame=single,
  columns=flexible,
  keywordstyle=\bfseries\color{blue!70!black},
  commentstyle=\itshape\color{green!40!black},
  language=Python,
  showstringspaces=false
}

% ## Glossaries ##
\usepackage{glossaries}
\makeglossaries

\newglossaryentry{dictionary}
{
    name=dictionary,
    description={When we refer to a dictionary, we mean a data structure which solves the dictionary problem~\cite{Wiki-Associative-Array}. This can often simply be thought of as a hash map}
}
\newglossaryentry{function}
{
    name=function,
    description={See \gls{functions}}
}
\newglossaryentry{functions}
{
    name=functions,
    description={When we refer to functions regarding the 'digital taskboard', we mean a row/ column label for the 2D data structure. This label is used to indicate data entries, which are functions/ tasks that need to be performed by the medical staff and are not to be confused with functions in the mathematical sense}
}

% Formalia goes here:
\newcommand\course{Master's Thesis} %<--- Kursus
\newcommand\datoen{F25} %<--- Dato
\newcommand\opgavetitel{A case study in: Digitalising the Danish healthcare sector} %<--- Navnet på opgaven


% Make GREAT looking header and footer:
\fancyhead[L]{Master's Thesis}
\fancyhead[R]{\rightmark}

\begin{document}
    \begin{titlepage}
    \raggedright
    \includegraphics[width=0.15 \textwidth]{figures/Titlepage/Coral_Red_RGB.pdf}
    \hfill
    \includegraphics[width=0.45\textwidth]{figures/Titlepage/tex_dtu_compute_a_uk.pdf}
    \begin{center}
        \vspace{3 cm}
        \hrule
        \vspace{.3cm}
        { \huge {\bfseries {\opgavetitel}}
        } 
        % Title of the report:
        %The impact of regularization on classifiers and explaining the explainability of axiomatic attributions
        
        \vspace{.1cm}
        { \LARGE {\bfseries 
            {
                Master of Science in Engineering
                \\
                Human-Centered Artificial Intelligence
            }
        }
        }
        \vspace{.5cm}
        
        \hrule
        \vspace{1.5cm}
        
        \textbf{Authors}\\
        \vspace{.5cm}
        \centering
        
        % add your name here
        Felix Bo Caspersen | s183319\\
        
        \vspace{1.5cm}
        
        \textbf{Supervised by}\\
        \vspace{.5cm}
        \centering
        
        Professor Aasa Feragen \\
        \vspace{1.5cm}
        
        Kongens Lyngby \\
        \centering \datoen % Dags dato
    \end{center}
\end{titlepage}
    \begin{abstract} 
    Denmark's healthcare sector faces mounting pressure from an ageing population, staff shortages, and expanding administrative demands. In response, this thesis explores how targeted digital tools can alleviate some of the everyday burdens placed on clinical personnel. First, we investigate the process of digitalising a daily whiteboard - used to coordinate tasks - by merging data from multiple, often incompatible sources. Here, the main challenge was to parse and reconcile differing formats, exposing deeper systemic problems in data integration.
    \\
    Next, we propose a scheduling assistant tool employing constraint programming to automate the allocation of staff across tasks, subject to intricate real-world rules. While our prototypes highlight the feasibility and potential efficiency gains of such an approach, they also underscore the need for robust data handling and a user-friendly interface that bridges informal requirements and formal logic. Finally, we outline an automated schedule validator intended to verify whether final schedules remain faithful to overarching managerial plans.
    \\
    Collectively, these contributions illustrate the promise of digital solutions in streamlining clinical workflows and the practical difficulties - ranging from entrenched practices to isolated IT systems - that hamper meaningful progress. Despite the partial nature of our prototypes, our work indicates that combining methodical constraint modelling with human-centred design can yield functioning strategies for modernising administrative routines in the Danish healthcare sector.
\end{abstract}
    \section*{Preface and Acknowledgements}
This project is written as part of the Master of Science in Human-Centered Artificial Intelligence at the Technical University of Denmark, Department of Applied Mathematics and Computer Science. Professor Aasa Feragen supervised this project. The project was completed from the 2nd of September 2024 to the 9th of March 2025. All the code supporting the project is in the GitHub repository \github{https://github.com/WeatherBoy/MastersProject_Hospital_Services}\footnote{For completeness: \href{https://github.com/WeatherBoy/MastersProject_Hospital_Services}{https://github.com/WeatherBoy/MastersProject\_Hospital\_Services}.}.
\\
\\
I want to thank Aasa Feragen for her guidance in the project's overall direction. Furthermore, I owe a huge debt of gratitude to Annie Gam-Pedersen for not only acting as a liaison between me and OUH during the entire project but for without whom this project would never have been more than a questionnaire and a hope. Another huge thanks goes to the OUH staff, particularly Katrine and Rikke, for enduring my shenanigans.
I would also like to thank Malthe \O rberg Gedsted and Tobias Drue for their tenacity in providing mission-critical feedback throughout the entire writing process.
Lastly, I would like to give my sincerest thanks to Tobias Drue, Rasmus Molsted and Jakob Kammeyer, for without whom and their unyielding support, I would almost assuredly have succumbed to despair.

    \setcounter{page}{1}
    \pagenumbering{roman}
    \tableofcontents
    \newpage
    \printglossaries
    \newpage

    \setcounter{page}{1}
    \pagenumbering{arabic}
    \section{Introduction}
The Danish healthcare sector is characterised by highly skilled medical professionals operating under constant and evergrowing time pressure.
Denmark's population is increasingly ageing, with a trend toward an older demographic. Statistics Denmark~\cite{dst-older-pop} states that the number of Danish citizens aged $80$ and above is expected to increase by $43$\% within the next decade, while the group aged $60$-$79$ is also expected to grow by $6.8$\%\footnote{The numerical increase is from $319,874$, in $2024$, to $457,488$, in $2034$, for $80$+ and $1,279,671$, in $2024$, to $1,359,430$, in $2034$ for $60$-$79$}. Moreover, recruiting qualified healthcare personnel is becoming increasingly difficult. The Ministry of Higher Education and Science~\cite{MoHEaS-healthcare-rectruitment} reports that, although there has been a recent increase in the number of students graduating in healthcare-related fields, this is still insufficient, and the lack of qualified healthcare personnel is only expected to worsen over time. The Ministry of Finance~\cite{MoHEaS-healthcare-rectruitment} anticipates that the healthcare sector will lack approximately $15,000$ social and healthcare workers by $2035$, and in response, the government plans to partner with developing countries\footnote{Mentioned are India and the Phillipines.} to help recruit healthcare personnel.
\\
\\
Over the past three decades\footnote{See \autoref{fig:healthcare_digitalisation}, 'Danske Regioner' made a pleasant visualisation.}, digital initiatives and platforms have been developed in an attempt to alleviate the challenges posed by an understaffed healthcare sector by enhancing workflow efficiency and patient care. However, due to the complexity within the healthcare sector arising from diverse patient needs, intricate protocols and procedures, and a plethora of administrative tasks, the resulting systems put in place\footnote{The classic example here being "Sundhedsplatformen", Epic's healthcare-IT system, that since its inception in 2016, has been under constant critique~\cite{DR-healthcare-platform, Altinget-healthcare-platform, TV2-healthcare-platform}. Though, in all honesty, in 2021, in a user survey by "Center for Patientinddragelse"~\cite{SP-user-survey} they noticed a slight uptick in general satisfaction amongst users, i.e. the healthcare staff.} are often challenging for healthcare personnel to use effectively. While the initiative behind these solutions is commendable, without careful attention to human-centred design and usability, even the most technologically advanced solution risks going unused, ultimately failing to support the healthcare personnel it was designed to assist.
\\
\\
This thesis emerges from a desire to address the real-world, everyday digital challenges faced by healthcare personnel. We approached this by first sending out a questionnaire. We did this in an attempt to map the current state of the Danish healthcare sector and assess whether there would be any trends indicating general national issues.
\\
Initially, the hope was to broaden our understanding of modern techniques such as Large Language Models (LLMs) in order to streamline the documentation process. However, first-hand contact with the sector revealed higher priority problems. Thus, we decided that the focus should be on more fundamental digital tools rather than pursuing a complex, theoretically appealing AI solution. These were threefold:
\begin{enumerate}
    \item Digitalising a taskboard system, which currently requires manual data transcription by a nurse at the start of each day.
    \item Developing a scheduling assistance tool to help administrative staff assign personnel while adhering to various practical constraints.
    \item Automating the validation process needed to check the final schedule against management's desired plan.
\end{enumerate}
None of these tasks proved as straightforward as we had anticipated. The initial concept for the taskboard digitalisation suggested a relatively simple data-integration project. However, the implementation process ultimately extended over several months due to unforeseen complexities in data formats, infrastructure limitations, evolving requirements and more.


    \section{Theory}
Of the three main tasks explored in this thesis, developing a \textit{scheduling assistance tool}, which we henceforth will refer to as '\textit{the scheduling task}', required the most mathematical approach. \textit{The scheduling task} is a case of the \textbf{Generalised Assignment Problem}\cite{Wiki-general-assignment-prob}, which can ultimately be expressed as a \textbf{Linear Program}\cite{Wiki-linear-programming}, specifically, a \textbf{Binary Integer Program}.
The following sections explore different theoretical approaches that might yield a satisfactory solution to such a \textbf{Generalised Assignment Problem}.

\subsection{Genetic Algorithms (GAs)}
A Genetic Algorithm, as first introduced by John Holland\cite{Genetic-Algorithm-original} in 1975\footnote{Though later revised in 1992.}, is a metaheuristic within primarily \textbf{Computer Science} and \textbf{Operations Research} (OR). Inspired by evolutionary theory's natural selection, a GA is typically employed for \textbf{Optimisation} and \textbf{Search problems}.
\\
In short, inspired by natural selection, a GA scheme initiates, at random, a pool of \textit{candidate solutions} called the \textit{initial generation}. It then evaluates the solutions with respect to the given domain and assigns each of them a \textit{fitness score}. Finally, it pairs the solutions with the best fitness, creating a new set of candidate solutions called the \textit{next generation}. However, in order to avoid converging towards some \textit{local optimum}, a bit of stochasticity (\textit{mutation}) is introduced in each generation. The GA scheme then iterates this entire process, stopping when reaching some criteria, either time or a threshold on the fitness score.
\\
\\
A Genetic Algorithm, boiled down to its base parts, consists of an objective function $f : \mathbb{R}^N \rightarrow [0,1]$, for some $N \in \mathbb{N}_0$, a \textit{selection criteria}, and a population of \textit{phenotypes}, the candidate solutions - each with its own \textit{genotype}. They are domain-dependent but are frequently represented as an $N$-dimensional vector. The genotypes will often be the actions or set of rules from which the phenotype can be derived. However, this is also largely dependent on the domain. For some problems, the genotype and phenotype are simply identical.
\\
Finally, the objective function $f$ and selection criteria are largely the crux of the GA scheme. While the phenotypes and their respective genotypes, are essentially bound to the domain, the objective function and selection criteria can be modelled for better results. The objective function and selection criteria could be considered the Genetic Algorithm's counterpart to Machine Learning's (ML) hyperparameters.
\\
The objective function $f$, evaluates to a scalar, usually standardised to the real interval $[0,1]$, hence $f$ doesn't need to be differentiable nor even continuous. This feature can allow for some creative objective functions specifically suited to the given problem.
\\
The selection criteria is how we pick the parents for the next generations. The parents, a set of two phenotypes, always produce two children to keep the population size constant throughout the iterations of the GA scheme. However, picking the parents can be a science in itself. Detailed below are some different algorithms used to perform the selection step.

\subsubsection{Selection Algorithms}

    \section{Methodology}
\subsection{First Steps - The Questionnaire}
Rather than researching some obscure nook or cranny of mathematics, this project was initially motivated by a desire to try and implement 5 years of education in order to assist with some current real-world problems. From prior work, we hypothesised that there might be quite a handful of such problems within the Danish healthcare sector. As such, we laid the groundwork in March of 2024 by sending out a questionnaire, which can be seen in appendix \autoref{fig:questionnaire_healthcare_personnel}, to \textbf{Rigshospitalet}, \textbf{The Hospital of Hvidovre}, \textbf{The regional Hospital of Northern Jutland}, \textbf{The University Hospital of Aalborg}, \textbf{Hospital unit Midt}, \textbf{The Hospital of Holbæk} and \textbf{The department of orthopaedic surgery at the University Hospital of Odense}. This was met with quite varying responses. Most deemed that answering such a questionnaire would be an abhorrent misuse of their personnel's limited time. This, admittedly, left us a bit discouraged. However, we still received a total of 39 answers from medical professionals.
\\
The questionnaire consisted of two questions\footnote{With the option to add your name and/or mail for possible further inquiry.}:
\begin{itemize}
    \item What is the biggest and most frequent problem in your daily life as a healthcare professional?
    \item What do you think is the biggest and most frequent problem in the daily life of your patients?
\end{itemize}
The responses about the daily life and tasks of healthcare professionals varied significantly, but they could likely be summarised as a general dissatisfaction with the understaffed and underfunded state of the healthcare sector.
\\
Many complained about the intricate journaling processes - documenting the patients' treatment pathways, reading through a lot of unnecessary documentation and documenting the same thing twice. 
\\
There was a plethora of replies criticising the IT systems currently in place. Complaining that they either didn't work, were too slow or kept crashing. Additionally, one respondee noted that they have a lot of programs that advertise useful functionality, but because they aren't integrated with all the other individual solutions, the healthcare staff end up having to open multiple programs documenting the same data many times over.
\\
Some mentioned that there are a lot of wrongful queries to the emergency line that have to be redirected. It was also added that a filter or automatic redirection might save a lot of critical time.
\\
Finally, the vast majority of responses were aimed at a lack of resources. This included being understaffed, resulting in insufficient time to care for patients and individual staff members being assigned too many tasks. Additionally, there was a shortage of critical inventory, such as bed space and essential medical equipment. Moreover, the existing equipment was often old or outdated due to budget cuts.
\\
\\
The responses regarding the everyday problems of patients weren't nearly as varied. These were mainly regarding long waiting times and a lack of transparency. Out of 39 responses, 12 mentioned that patients had expressed concerns about unclear treatment pathways. Additionally, 18 out of 39 responses indicated that patients had complained about the length of the waiting times. Lastly, a handful of respondents (approximately 6) mentioned that their patients feel overlooked, which may also indicate an understaffed healthcare sector.
\\
\\
After reading the responses to our questionnaire and working in tandem with the healthcare sector for roughly six months, it has become undoubtedly apparent that the healthcare sector is in dire need of a digital revolution. A lot of tasks could be expedited seamlessly without contributing additional burdens to the medical personnel if only proper digital solutions were implemented. However, we would have been unable to work on any practical application within the healthcare sector if all responses we had received were equally uninclined to collaborate. Therefore, we were overjoyed when the University Hospital of Odense (OUH) reached out with an incredibly accommodating response.

\subsection{The Digital Taskboard}
OUH responded that their department of Orthopaedic surgery uses a whiteboard as a daily taskboard. The taskboard details which \textbf{tasks} (referred to throughout the rest of the section as \gls{functions}) are performed by \textbf{who}, in what \textbf{timeslot}, \textbf{where}, and any \textbf{extra} information if applicable. A picture of the current whiteboard is shown in \autoref{fig:Whiteboard-Taskboard}. However, this whiteboard is being drawn up daily by manually transcribing data from two distinct spreadsheets. OUH  informed us that the process usually takes 10-20 minutes, time that could be better spent on more critical tasks by the clinical staff, and as such, they wished for a digital replacement for the whiteboard. Therefore, this thesis began with the seemingly straightforward task of merging two 2D datasets and visualising them on a screen comparable to a whiteboard.
\\
For clarity, we would like to note that the two spreadsheets are separated purely for administrative purposes; the separation doesn't add any additional dimensionality. Hence, it can still be displayed without loss of dimensionality on a plane like a whiteboard or screen. 

\begin{figure}[H]
    \centering
    \includegraphics[width=0.9\textwidth]{figures/Methodology/Whiteboard-Taskboard-Redacted.jpg}
    \caption{Whiteboard taskboard}
    \small
    \raggedright 
    A picture of the taskboard currently in place at OUH. Names and tasks have been redacted to preserve the anonymity of OUH's personnel. 
    \label{fig:Whiteboard-Taskboard}
\end{figure}

The two spreadsheets, which need to be merged in order to arrive at the correct data, consisted of a \emph{ward assignment chart}, in the form of a Microsoft Excel file and a web application made by a third party called HosInfo. A snippet of the ward assignment chart can be seen in \autoref{fig:wac-snippet}, and a snippet of the HosInfo interface can be seen in \autoref{fig:HosInfo-snippet}.
\\
The immediate issue was that the HosInfo data was inaccessible. The HosInfo application (which we will refer to onwards as just '\emph{HosInfo}') didn't have an API, and it only has two access levels: surface-level access for viewing the data and full administrative control. As such, our only option was to scrape the HTML pages from HosInfo. Scraping was a wholly new endeavour for us, so at first, we tried using Python's inherent \texttt{requests} package. Here, we injected \emph{cookies} and a \emph{header} containing our login information. However, it would soon prove that what seemed to be a bug at first, was actually just not an automatic solution at all. The injected information would have to be updated regularly with a manual login for this approach to work. Therefore, we made a switch to the open-source, third-party tool \texttt{selenium}. \texttt{selenium} works by instantiating a WebDriver, which can automatically drive the browser. In short, this meant that we gave it instructions to navigate separately to the three specified text elements for login input and fed it our credentials. In \autoref{fig:HosInfo-login}, a screengrab of the HosInfo login gateway is shown. After feeding the login credentials, the WebDriver manually navigated to the login button and clicked it. This prompts a load before transitioning from the login page to the correct URL from where we could scrape the data, so we injected a manual waiting time of at most ten seconds until a unique element appeared. We simply chose a unique element at random and used its XPATH in the HTML code for identification. This is all to say that the webscraping solution is incredibly fragile and highly prone to the slightest changes made to the HosInfo application.

\begin{figure}[H]
    \centering
    \includegraphics[width=0.95\textwidth]{figures/Methodology/Ward-Assignment-Chart-Redacted.png}
    \caption{Ward assignment chart}
    \small
    \raggedright 
    A snippet of the ward assignment chart used for bookeeping weekly tasks. Names and tasks have been redacted to preserve the anonymity of OUH's personnel. 
    \label{fig:wac-snippet}
\end{figure}

\begin{figure}[H]
    \centering
    \includegraphics[width=0.95\textwidth]{figures/Methodology/HosInfo-Redacted.jpg}
    \caption{HosInfo interface}
    \small
    \raggedright 
    A snippet of the HosInfo interface used for storing planning information at OUH. Names and tasks have been redacted to preserve the anonymity of OUH's personnel. 
    \label{fig:HosInfo-snippet}
\end{figure}

By submitting our credentials and automatically navigating to the correct webpage, we could now extract it in full as an HTML file. Here, we opted for using the very standard solution of parsing the HTML file with the \texttt{bs4} package, more commonly known as \texttt{BeautifulSoup}. Parsing the HTML was, to our great luck, quite easy. Although the spreadsheet isn't formatted as an HTML \texttt{table}, every single cell (or entry) was the same division (\texttt{div}) with the class name: \emph{"single-description"}. Likewise, were the \gls{functions} (the row labels) but instead with the class name: \emph{"single-function"}. As there were seven columns, one for each day of the week, we could iterate through our 1D array of entries but retain the 2D order by using the row indices:

\begin{equation*}
    \text{row\_index} = \text{entry\_index} \mod 7
\end{equation*}
and the column indices:
\begin{equation*}
    \text{column\_index} = \left\lfloor
        \frac{\text{entry\_index}}{7}
    \right\rfloor.
\end{equation*}
It was helpful to keep the index as some \gls{functions} (rows) contained irrelevant information, and others remained unchanged from week to week but introduced significant bugs when we tried to include them. An example of rows that weren't relevant was the large entries describing that all units (agents) either were taking a course, had a planned vacation or another kind of time off, or that the agents were ill. When we initially tried to exclude some of these rows, we did it by their index, keeping a \gls{dictionary} of the \gls{function} label to index translation. Unfortunately, the indices didn't stay fixed and could change from week to week. This also introduced a large bug to our system, which we attempted to alleviate by simply skipping rows based on their label instead. The idea behind using indices rather than labels was a hope that, although the \gls{function} label might be renamed, it would retain its index. Ideally, the solution wouldn't have to skip any rows, but seeing as that wasn't an option, our program was left vulnerable to \gls{function} label renaming.

\begin{figure}[H]
    \centering
    \includegraphics[width=0.55\textwidth]{figures/Methodology/HosInfo-Login.png}
    \caption{HosInfo login gateway}
    \small
    \raggedright 
    The login gateway for the HosInfo application.
    \label{fig:HosInfo-login}
\end{figure}

Now, with all the preprocessing of the data, we had gone from a webpage with some entries in the form of a table to a full HTML file and all the way to a manageable 2D table (a \texttt{pandas} table in Python) with string entries and row labels. This brings us to the second glaring issue: HosInfo doesn't enforce any formatting of its entries. As such, we had to somehow arrive at a pattern for all the string entries constituting our data.
\\
In \autoref{tab:HosInfo-example-entries}, we give an example of some of the possible entries in HosInfo. Some of the formatting is extremely simple to spot for the human eye, and for a non-technically inclined individual, some of these entries probably read just fine. However, there is still a lot of ambiguity as to how it should be interpreted. In \autoref{tab:HosInfo-good-processing}, we give an example of what could probably be considered a good interpretation, and in \autoref{tab:HosInfo-poor-processing}, we provide an example of some of all the possible pitfalls a computer program might be prone to stumble into.  

\begin{table}[H]
    \centering
    \begin{tabular}{|l|}
        \hline
        HARRYP (08:15 - 15:47) spells                                                                                                                                                          \\ \hline
        \begin{tabular}[c]{@{}l@{}}
            DORA\_TE (07:30 - 14:25)\\
            PAUL\_A (07:00 - 12:00) (12:00 - 15:00) Hr. 8-12
        \end{tabular}                                                                    
        \\ \hline
        \begin{tabular}[c]{@{}l@{}}
            DAENERYS\_T (08:00 - 16:00) stormborn\\
            STEVEN\_STUD1 (08:00 - 16:00)
        \end{tabular}
        \\ \hline
        \begin{tabular}[c]{@{}l@{}}
            THOMAS\_TT (07:30 - 15:47)\\
            WALTERHW (08:00 - 16:12) chemistry\\
            JESSIE\_P (07:30 - 15:30) 13-15\\
            LEAH\_OS (07:00 - 15:43) Hr. 07:00 - 13:00
        \end{tabular}
        \\ \hline
        \begin{tabular}[c]{@{}l@{}}
            KATARA\_OTNWT (07:00 - 15:30)\\
            SOKKA\_STUDENT (07:00 - 15:30)
        \end{tabular}
        \\ \hline
        GERALT\_OR   
        \\ \hline   
    \end{tabular}
    \caption{HosInfo example entries}
    \label{tab:HosInfo-example-entries}
\end{table}

\begin{table}[H]
    \centering
    \begin{tabular}{|l|l|l|}
        \hline
        \textbf{Name}         & \textbf{Time} & \textbf{Extra}  \\ \hline
        Harry P.              & 08:15 - 15:47 & spells          \\ \hline
        Dora T. E.            & 07:30 - 14:25 &                 \\ \hline
        Paul A.               & 07:00 - 15:00 & (08:00 - 12:00)          \\ \hline
        Daenerys T.           & 08:00 - 16:00 & stormborn       \\ \hline
        Steven (STUD1)        & 08:00 - 16:00 &                 \\ \hline
        Thomas T. T.          & 07:30 - 15:47 &                 \\ \hline
        Walter H. W.          & 08:00 - 16:12 & chemistry       \\ \hline
        Jessie P.             & 07:30 - 15:30 & (13:00 - 15:00)         \\ \hline
        Leah O. S.            & 07:00 - 15:43 & (07:00 - 13:00) \\ \hline
        Katara O. T. N. W. T. & 07:00 - 15:30 &                 \\ \hline
        Sokka (Student)       & 07:00 - 15:30 &                 \\ \hline
        Geralt O. R.          & -             &                 \\ \hline
    \end{tabular}
    \caption{HosInfo good processing}
    \label{tab:HosInfo-good-processing}
\end{table}

\begin{table}[H]
    \centering
    \begin{tabular}{|l|l|l|}
        \hline
        \textbf{Name}              & \textbf{Time} & \textbf{Extra}           \\ \hline
        Harryp                     & 08:15 - 15:47 & spells                   \\ \hline
        Dora T. E.                 & 07:30 - 14:25 &                          \\ \hline
        Paul A.                    & 07:00 - 12:00 & (12:00 - 15:00) Hr. 8-12 \\ \hline
        Daenerys T.                & 08:00 - 16:00 & stormborn                \\ \hline
        Steven S. T. U. D. 1.      & 08:00 - 16:00 &                          \\ \hline
        Thomas T. T.               & 07:30 - 15:47 &                          \\ \hline
        Walterhw                   & 08:00 - 16:12 & chemistry                \\ \hline
        Jessie P.                  & 07:30 - 15:30 & 13-15                    \\ \hline
        Leah O. S.                 & 07:00 - 15:43 & Hr. 07:00 - 13:00        \\ \hline
        Katara O. T. N. W. T.      & 07:00 - 15:30 &                          \\ \hline
        Sokka S. T. U. D. E. N. T. & 07:00 - 15:30 &                          \\ \hline
        Geralt O. R.               & -             &                          \\ \hline
    \end{tabular}
    \caption{HosInfo poor processing}
    \label{tab:HosInfo-poor-processing}
\end{table}

We loathe to admit it, but in order to have any fighting chance at formatting the strings, as shown in \autoref{tab:HosInfo-example-entries}, we elected to use regular expressions (regex). More precisely, we used the following regex pattern:
\begin{verbatim}
    ^([^\s(]+(?:_[^\s(]+)?)\s*(?:\(([^)]+)\))?\s*(.*)$
\end{verbatim}
This regex pattern essentially captures three groups. We will do our best to break it down. A good way to understand it is by dividing it into the following:
\begin{verbatim}
    ^
        ([^\s(]+(?:_[^\s(]+)?)   # (group 1)
        \s*
        (?:\(([^)]+)\))?         # (group 2, optional)
        \s*
        (.*)                     # (group 3)
    $
\end{verbatim}
Here, \verb|^| signifies the beginning of the string. \verb|([^\s(]+(?:_[^\s(]+)?)| captures \emph{group 1}. The first part, \verb|[^\s(]+|, captures "one or more characters that are not whitespace \verb|\s| and are not the character starting parenthesis \verb|(|"\footnote{This is the standard formulation employed for regular expressions.}. Then \verb|(?:_[^\s(]+)?| is a \emph{non-capturing} group, i.e. what it captures isn't a distinct group, but rather it adheres to a preexisting, in this case \emph{group 1}. It captures a group matched by the underscore character \verb|_| followed by "one or more non-whitespace, non-\verb|(| characters". However, this entire non-capturing group is optional, signified by the question mark character \verb|?|. The next line: \verb|\s*| is a standard expression which matches zero or more whitespace characters (effectively discarding them).
\\
Next, \verb|(?:\(([^)]+)\))?| is a second non-capturing group, which works in a somewhat peculiar way. It is also optional (again signified by the \verb|?|); however, if there is a closed parenthesis, given by a \verb|(| and a \verb|)| character, then this part of the expression \verb|([^)]+)| captures all characters which aren't the \verb|)| character, i.e. everything in the parenthesis. This is also followed by the standard: \verb|\s*|, discarding zero or more whitespace characters.
\\
Finally, what we have marked as \emph{group 3}: \verb|(.*)| simply captures everything else until the end of the string.
\\
In an attempt to explain it in a slightly more human fashion, we can classify the three groups as:
\begin{itemize}
	\item Group 1: Captures the name and the initials after the underscore, if present.
	\item Group 2: If a closed parenthesis exists, this captures its innards, i.e. this captures (the first) timeslot, if present.
	\item Group 3: This is anything after a potential closed parenthesis, or alternatively anything after the name, i.e. this captures any trailing appendages, e.g. this could be extra information as highlighted in \autoref{tab:HosInfo-good-processing}.
\end{itemize}
This way, we could separate forename from initials and a timeslot from other possible additional information. However, as showcased in \autoref{tab:HosInfo-poor-processing}, there are still a multitude of unsolved problems. Names that are contracted with their initials instead of separated by an underscore; multiple timeslots all encapsulated within closed parentheses; inconsistent formatting of timeslots; initials that are actually monikers instead of initials. We have managed to work around some of these, e.g. there weren't a lot of monikers. Hence, we simply elected to keep a \gls{dictionary} and treat it as a special case, but this would obviously also have to be regularly updated. Yet, sadly, \autoref{tab:HosInfo-poor-processing} better represents our final result than \autoref{tab:HosInfo-good-processing}.
\\
\\
Despite all the troubles it proved to try and process the HosInfo data, it, as previously alluded to,  still only tells half of the story. In order to show the complete picture and give the actual representation which OUH wanted, the HosInfo data needed to be merged with the data provided by a weekly ward assignment chart. The assignment chart was made with Microsoft Excel, so, in contrast to HosInfo, it was a piece of cake to simply load it into a \texttt{pandas} table, which could be manipulated in Python.
\\
Alas, here, the third glaring issue reared its head. Although easy to load into Python, merging the data from the assignment chart with the HosInfo data proved incredibly difficult. Firstly, the column at which the two 2D datasets should be compatible hadn't been given a standardised naming convention. This was the \gls{function} column; the one presented as rows at HosInfo and which, as can be seen from \autoref{fig:wac-snippet} is a column in the assignment chart, which indicates what personnel are performing which tasks. An example of this incompatible naming convention could be a task in the assignment chart called \emph{'SHOULDER KIDS'}, while at HosInfo, it might be called \emph{'SHOULDER 1'}. Other times, the task had attached a timeframe in the label at one place, while not at the other, e.g. \emph{'KNEES until 14'} and simply \emph{'KNEES'}. However, the names referred to the same, but it is our understanding that different departments, or maybe more sections, at OUH had individually elected to name them. It is, in turn, also our understanding that although the naming wasn't standardised, this didn't lead to discrepancies for the healthcare staff and to them, the translation was obvious. Sometimes, it might have been because a task was accompanied by a doctor who specialised in working with kids, and therefore, that doctor performing \emph{'SHOULDER 1'} must match with the nurse in HosInfo assigned to \emph{'SHOULDER KIDS'}. Other times, it might have been because they had simply grown accustomed to a special label meaning the same as an enumeration, e.g. that \emph{'SKH HAND'} and \emph{'HAND 2'} is a match\footnote{where \emph{SKH} is an abbreviation for the danish \emph{emergency}.}.
\\
The only solution is obviously a translation. We had expected that we would have to maintain a \gls{dictionary} over the translations; however, to our great pleasure, OUH decided to try and standardise the naming between these two spreadsheets.
\\
This resulted in at least a month-long ordeal of back-and-forth. We made a simple check that stored entries from each of the two spreadsheets if they didn't have a matching pair in the other spreadsheet. This way, in collaboration with the healthcare personnel at OUH, we could figure out which labels should be renamed and which labels were meant to not have a matching pair. Because, yes, some of the \gls{function} labels from both the assignment ward data and HosInfo data included critical information, but they didn't have a matching pair in the corresponding spreadsheet. Our immediate conclusion was that they were incompatible, but the more technical answer is that they should simply have their own column.

\begin{figure}[H]
    \centering
    \includegraphics[width=0.95\textwidth]{figures/Methodology/Final-Taskboard-Redacted.png}
    \caption{Final visualisation of the taskboard}
    \small
    \raggedright 
    The final version of our visualisation of the taskboard. Names and tasks have been redacted to preserve the anonymity of OUH's personnel. The columns translated to english are (from left to right): "Name", "\Gls{function}", "Ward", "Time", "Doctor" and "Additional notes". Note that in column two (the \Gls{function} column), the entries with a timeslot suffix are placed at the bottom of the taskboard. Additionally, as described, the "flex wards" are designated in the third column in parentheses.
    \label{fig:Final-Taskboard}
\end{figure}

The HosInfo data does not include information about which ward the \gls{function} is supposed to be performed at. However, as the name would imply, so does the ward assignment chart. Sometimes, an entry (a ward) in the assignment chart might be flagged with \emph{'FLEX'}. This means that that ward is "flexible", i.e. multiple people may come and go throughout the day. So if ward \(m\), in the assignment chart, is flagged with \emph{'FLEX \(n\)'}, then the agent \(a\) assigned to ward \(n\) may also be using ward \(m\) throughout the day. Therefore rather than just writing
\begin{center}
    "\(a\) is assigned to ward \(n\)",
\end{center}
a more accurate description would be
\begin{center}
    "\(a\) is assigned to ward \(n\), (flex ward \(m\))".
\end{center} 
Furthermore, a \gls{function} in HosInfo might have a timestamp or timeslot as a suffix. These were intentionally not meant to be paired with a ward from the assignment chart (unlike the rest of the \gls{functions}). However, they instead served as additional information, indicating some kind of special task to the healthcare staff\footnote{Admiteddly, at the time of writing, we have no clue what additional information this was intended to carry.}. OUH desired that these special functions with a timestamp all go to the bottom of the taskboard, so naturally, we complied.
\\
\\
What we have described up until now concludes the working functionality, which we have managed to incorporate within the final version of our visualisation of the taskboard. In \autoref{fig:Final-Taskboard}, we have provided one example of the culmination of these roughly three and a half months' worth of work.
\\
\\
After much deliberation and back-and-forth, it seemed as though we had arrived at an adequate solution. Alas, we ran into one final major roadblock, for which we have yet to find a reasonable solution.
\\
OUH naturally wished for the taskboard to have online update capabilities in case of possible illness or other reasons for urgent absence. Although true online\footnote{We would, maybe in all honesty not on the most substantial grounds, argue that for an algorithm to be considered truly online, it would have to update in response to a change.} update capabilities seemed infeasible; in theory, employing a scheme which mimics online updating would be quite straightforward. This would require a server which could run at set intervals, say every 5 minutes, throughout the working hours, and in that way, it could reprocess the two 2D datasets and capture any possible updates. It seemed as though OUH was willing to try out the server solution, yet we never made it that far, which brings us to the final roadblock.
\\
HosInfo contains information on \textbf{who} performs which \textbf{\gls{function}} within which \textbf{timeslot} and any possible additional \textbf{notes}. The ward assignment chart contains a lot of the same information. The timeslots and additional information are missing, but this is alleviated by merging with the HosInfo data. The key issue is that all the healthcare staff are indicated only by their initials in the assignment chart. The crucial detail is that absence is only marked in the assignment chart and not updated in the HosInfo data. As exemplified by \autoref{tab:HosInfo-example-entries} and \autoref{tab:HosInfo-good-processing}, we use the full first names and mark last names by their initials. Although, due to some poor architectural decisions on our behalf,  it would be a non-insignificant refactor of our codebase to extract the name from the assignment chart rather than HosInfo; in order to provide online updates, we would have to retain a \gls{dictionary} of translations from the healthcare personnel's initials to their actual names. This was the final nail in the coffin. The solution is obviously still feasible, but at this point, we deemed that we had given the digital taskboard our best shot, and instead, we attempted to move on to another problem.

\subsection{The Scheduling Assistant Tool}
As unclear it was just how intricate of a task it would be to merge two 2D datasets, as equally clear was it that, on its own, it couldn't provide sufficient academic foundation for a Master's thesis in artificial intelligence.

    \section{Analysis}
\subsection{Identifying the Problem}
A good part of this project was dedicated merely to identifying what kind of problem the scheduling task posed. While working on the digital taskboard, this issue used up all of our idle "brain bandwidth". We spent the entire first month pondering commonly accepted AI optimisation techniques, such as the GA scheme. However, we kept running into dead ends. Namely, it seemed incredibly difficult to construct a suitable objective function that could teach such optimisation techniques, the underlying rules adhering naturally to the problem. After some time, it felt like designing a black box to learn the rules we already knew rather than employing them to find a solution. Thus, we came to the conclusion that we had, in all likelihood, gone about this problem in the wrong manner. As such, we had no other option but to do some research.

\subsubsection*{The Scheduling rabbit hole}
If the reader, like us, isn't very well-versed in schedule planning and the field of Operations Research~\cite{Wiki-Operations-Research}, they might also had been lured into the \emph{Scheduling}~\cite{Wiki-Scheduling-computing, Wiki-Optimal-job-scheduling, Wiki-Job-shop-scheduling} rabbit hole. Scheduling is a discipline that the vast majority of us enjoy on a regular basis, as it is used predominantly in CPUs. However, it is an entirely different algorithmic discipline for assigning jobs to machines. In this case, each job has a runtime, and sometimes the runtime varies based on the machine. At first, we thought this might just be an abstraction that we could still apply to our problem, but unfortunately, it isn't so simple. Notice that 'Scheduling', as just described, is a discipline within computer science and differs from our problem, which we elected to call 'the scheduling task'.
\\
One might think of the scheduling task as a mosaic of \(i \times j\) squares in a grid, where each square can be one of \(n\) different colours. We have \(m\) stacks of tiles that can assume all \(n\) colours, but each is a distinct shape. However, we can't change the colours of the tiles we have, and for a tile to fit onto the mosaic, it must match the colour of the corresponding square. The contractor has told us that we can't have more than one of each shape for each row, or it will ruin the 'synergy' (or whatever). We want to see whether we can lay the entire mosaic using no more than the tiles we have at our disposal. However, there might be more constraints; for example, the contractor might not like to have more than two of the same shape adjacent to each other in a column.
\begin{figure}[H]
    \centering
    \begin{subfigure}[t]{0.73\textwidth}
        \centering
        \includegraphics[width=\textwidth]{figures/Analysis/BSAT-illustrated01.png}
        \subcaption*{An example mosaic in a grid (right) with available tiles (left).}
    \end{subfigure}
    \hfill
    \begin{subfigure}[t]{0.225\textwidth}
        \centering
        \includegraphics[width=\textwidth]{figures/Analysis/BSAT-illustrated02.png}
        \subcaption*{Succesfully laid mosaic.}
    \end{subfigure}
    \caption{Illustration of the scheduling task}
    \label{fig:the-scheduling-task-illustrated}
\end{figure}
On the other hand, we could think of Scheduling as having the weirdest mosaic ever. The contractor wants \(n\) rows; it doesn't matter what colours they have. The goal is to place our tiles in such a manner that the row which is furthest from the left wall is as close to the left wall as possible. However, we have to place all of our tiles. But they are annoying, magical tiles that change length according to which row we place them in.
\begin{figure}[H]
    \centering
    \begin{subfigure}[t]{0.715\textwidth}
        \centering
        \includegraphics[width=\textwidth]{figures/Analysis/Scheduling-illustrated01.png}
        \subcaption*{An example weird-mosaic (right) with available magical tiles (left).}
    \end{subfigure}
    \hfill
    \begin{subfigure}[t]{0.25\textwidth}
        \centering
        \includegraphics[width=\textwidth]{figures/Analysis/Scheduling-illustrated02.png}
        \subcaption*{Succesfully laid weird-mosaic. The orange dotted line indicates the length of the longest row (from the left wall).}
    \end{subfigure}
    \caption{Illustration of Scheduling}
    \label{fig:scheduling-illustrated}
\end{figure}
The scheduling task and Scheduling, as illustated\footnote{We would like to apologise for these two illustrations being, by no means, colourblind-friendly. We hope that the description will suffice instead.} in \autoref{fig:the-scheduling-task-illustrated} and \autoref{fig:scheduling-illustrated} respectively\footnote{We want to state very explicitly that we in no way claim this is the optimal solution to the Scheduling problem we presented; it is merely meant as an illustration.}, are clearly related. They might even be cousins. But they could never swap social IDs for a month without anybody noticing.

\subsubsection*{From Assignment Problem to Binary ILP to BSAT}
After a little more research, we found out that, what we had na\"ively translated from Danish as, \emph{scheduling} seemed to be modelled by the \textbf{Assignment Problem}~\cite{Wiki-assignment-prob}. The Assignment Problem can be defined as the following: Given \(N\) agents and \(N\) tasks, where the cost of assigning agent \(i\) to task \(j\) is \(c_{ij}\), each agent must be assigned one and only one unique task. Any agent can be assigned to any task and the goal is to find an assignment which minimises the total cost of all assignments,
\begin{equation*}
    \min \sum_{i}^N \sum_{j}^N c_{ij} \cdot x_{ij},
\end{equation*}
where \(x_{ij}\) is \(1\) if agent \(i\) is assigned to task \(j\), and \(0\) otherwise.
\\
This looked promising. There are algorithmic solutions with polynomial runtimes, and for all purposes, the costs can be modelled as identical. There is the issue, as we mentioned in Methodology (see \autoref{sec:scheduling-assistant}), that for \(N\) tasks and \(M\) agents, we usually have \(M \ll N\). However, we deemed this could be alleviated by solving each day individually. Then, if there was a day with more agents than tasks, we could simply substitute them with dummy tasks, which we could remove from our final solution. This would obviously add the \(T\) coefficient to a polynomial runtime, but it might still be faster than exponential for problems with many variables. It quickly became apparent, though, that for constraints posed by tasks such as the "Rygvagt" (see \autoref{sec:translating-constraints}), the solution would be infeasible at worst or, at best, non-polynomial.
\\
\\
However, we had garnered from The Assignment Problem that it could be written up as a \textbf{Linear Program}\cite{OR-Intro-Book} (LP). Our problem, the scheduling task, would be a case of a Binary Integer Linear Program (Binary ILP). This is because all of our decision variables are boolean; either an agent is assigned to a given task, or they are not.
Now, a binary ILP takes the form\cite{Integer-Programming-Book} 
\begin{equation}\label{eq:binary-ILP}
    \begin{aligned}
    \max \quad &\sum_{i = 1}^m \sum_{j = 1}^n c_{ij} \cdot x_{ij}
    \\
    \text{subject to} \quad &\sum_{j = 1}^n a_{ij} \cdot x_{ij} \leq b \qquad &\text{for} \quad i = 1, \hdots, m
    \\
    &\sum_{i = 1}^m x_{ij} \leq 1 \qquad &\text{for} \quad j = 1, \hdots, n
    \\
    \text{for} \quad &x \in \{0, 1\}^{m \times n},
    \end{aligned}
\end{equation}
where \(x\) is the binary decision variable, \(c_{ij}\) is the value, or cost, connected with making the decision represented by \( (i,j) \), \( a_{ij} \) is the penalty incurred for \( (i,j) \) and finally \(b\) is an upper bound on our cumulated penalty.
\\
For the small subproblem we worked on, we didn't have any non-linear constraints. Therefore, there is no technical reason preventing us from writing up our problem as presented in \autoref{eq:binary-ILP} and solving it using OR. But looking at OR led us to CP, and the formulation presented in \autoref{eq:binary-ILP} gives way to a Boolean Satisfiability Problem (BSAT), which is a subfield of CP. We can attempt to solve a BSAT using a satisfiability (SAT) solver.
\\
There are many advantages to using the BSAT formulation over the Binary ILP. For one, OR is still inherently an optimisation technique that solves problems in the same ballpark as Scheduling and The Assignment Problem. It isn't designed for this kind of problem, where we aren't interested in a solution that is \(99.998\%\) optimal, but rather where we first and foremost need a valid solution, optimisation second. As such, formulating the problem is a tedious endeavour. It grows wildly as a function of variables and constraints, and while the same can be said for the BSAT formulation, SAT solvers are built for this; they prune the search space, eliminating redundancy while seeking valid solutions. Additionally, as shown in \autoref{sec:translating-constraints}, modern libraries enabling SAT solvers, such as Google's OR-Tools, allow for straightforward handling of more complex constraints, which would be a nightmare to define for a Binary ILP.
\\
\\
As a last remark on the technical implementation aspect, we will quickly address the running time. As exemplified in \autoref{eq:decision-variable-form}, we will have a decision variable for each agent \(a \in [0, M]\), for each day \(i \in [0, T]\), for each task \(j \in [0, N]\). This is a nice representation, but a better and more straightforward definition is that we have a sum of \(N\) tasks over all \(T\) days. That way, we can readily define the total number of decision variables in our boolean formula as \(M \cdot N\) without having to deal with having an unequal amount of tasks for each respective day. As we have stated many times, boolean satisfiability is NP-complete. Therefore, we ultimately end up with a running time of \(2^{M \cdot N}\).
\\
\\
As for The Automated Schedule Validator, as aforementioned, we are convinced that it is technically feasible. Additionally, it would seemingly fit right in with the rest of the suite of disconnected software applications; as such, it seems viable. However, we cannot help but question whether it would be a good solution or if we would just be another in a long line of software implementations that had insisted they keep on their floaties - rather than ever letting them try to swim on their own.
    \section{Discussion}
We started this project by agreeing to provide OUH with a digital replacement for their current, manually transcribed taskboard (as detailed in \autoref{sec:Digital-Taskboard}). While true that this problem, on its own, couldn't provide a sufficient academic foundation for a Master's thesis in AI, we nevertheless made the grave error of underestimating just how cumbersome\footnote{Which, to her great credit, our supervisor had actually warned us about.} of a task digitalising the taskboard could be.
\\
\\
At a glance, digitalising the taskboard seemed no more complex than merging two 2D datasets, and we are still convinced it could have been were it not for OUH's currently in-place payroll and calendar overview system, HosInfo, serving as an ever-persistent thorn in our side. While the HosInfo implementation could be criticised plentily for its sluggish and featureless backend\footnote{Which we admittedly never had access to but only have seen in parsing when been guided through the system by the OUH staff.}, beyond the shadow of a doubt, our biggest gripe with it was its data handling or, rather, lack thereof. Not only does HosInfo make OUH's data nearly inaccessible by storing text data as pictures, but they also enforce no data formatting on their users. Additionally, rather than designing a solution capable of serving the greater planning effort at OUH, they have delivered a solution to a smaller subproblem while completely disregarding even the slightest notion of compatibility with the rest of OUH's IT environment.
\\
Such software makes it nigh impossible for OUH, not even to mention other external providers such as ourselves, to develop other solutions that need access to their data without forcing the end user to document everything many times over.
\\ 
However, HosInfo isn't the only third-party provider that is guilty of contributing to OUH's messy, tangled, and, at the same time, disconnected IT stack. We can only speculate on the reasons behind how it ends up as such. It might be due to the Danish and European procurement legislation~\cite{Udbudsloven, EU-Procurement-Legislation}, or in part due to a lack of technical insight at OUH, which limits them in seeing the greater scope of their administrative challenges. Nevertheless, it culminates in a system comprising a slew of individual IT solutions, which can't share data nor benefit from what each provides.
\\
Trying to work around this constant impediment is perfectly summarised by this single response to our questionnaire, which loosely translated into English would be:
\begin{center}
 \textit{We have many systems that can do quite a lot, but not always in conjunction with each other, and most of the time, you have to work across several applications.}
\end{center}
While (our interpretation of) this statement perfectly encapsulates our immense frustration after six months of efforts of working with (or around) the Danish healthcare sector's IT, we can still scarcely imagine having to work with it on a daily basis. Therefore, we thoroughly believe that a more cohesive and all-encompassing platform would be the right move.
\\
We envision a cloud-based platform with data formatting requirements which can serve data to the correct users. Such a platform should have a single standardised user interface (UI) and allow the administrative workers to extract and log data for their corresponding hospitals (or other clinical facilities). It should facilitate an API allowing approved independent third-party providers to create individual software solutions that could work off of this platform. Such a platform would immediately eliminate any need to log data multiple times, as it would all be gathered in one place. Furthermore, if such a platform enforced a UI design guide onto third-party providers and required them to implement their solution onto the platform, it would save the healthcare personnel from having multiple credentials and remembering different layouts for each and every single individual application.
\\
This idea would require a lot from the IT sectors of the Danish healthcare sector; in turn, we also hope that it would be easier to maintain if all of it was gathered onto a single platform, with the single uniform goal of facilitating more seamless implementation of IT solutions.
\\
However, this isn't exactly 'new tech', as "Danske Regioner" in 2024 proposed a 'Danish Healthcare Cloud'~\cite{Den-Reg-digitalisation}. The referenced note touches upon many of the same topics as we have just detailed. Rather than feeling amiss for only arriving at the same conclusion as somebody else had found way before us, we will instead let it enforce our belief that such a restructuring wouldn't simply be an improvement; it is a requirement.
\\
\\
This brings us nicely along to our next point, for although, in all likelihood, we wouldn't be qualified to implement such a platform, we did attempt to build The Scheduling Assistant tool, which we believe could have been a good starting point for OUH, at least.
\\
As mentioned, we didn't believe digitalising the OUH taskboard would be academically sufficient; therefore, we also pursued the scheduling task. We believed that we could implement this side-by-side with the digital taskboard. While we still think it technically feasible, as it has been done before~\cite{you-but-better} (and to a way greater extent), much like for the digital taskboard, we ran into multiple unforeseen difficulties.
    \section{Conclusion}
For this project, we set out to address the daily real-world challenges clinicians in the Danish healthcare sector face. We sought to accomplish this through three different projects: digitalising an essential taskboard at \acrshort{ouh}, developing a scheduling assistant tool, and an automatic schedule validator. While we managed to produce prototypes for merging differing datasets and modelling a scheduling task, we continually faced technical, bureaucratic, and human-centred challenges, which all serve to emphasise how real progress depends on more than solid code alone. Our experience suggests that, for any new digital solution to succeed in the Danish healthcare sector, key factors, such as GDPR compliance, IT integration, and end-user satisfaction, must be thoroughly considered from the very onset. Nevertheless, we remain convinced that automating these daily administrative tasks would not only be feasible but highly beneficial if the sector was willing to commit to clarifying constraints, standardising and ensuring the formatting of data, not to mention allowing enough time for new technology to be integrated with existing routines.

    \listoffigures
    \newpage

    \listoftables
    \newpage

    \section{Software versions used}
\begin{itemize}
    \item \texttt{beautifulsoup4}: v4.12.3
    \item \texttt{pandas}: v2.2.2
    \item \texttt{selenium}: v4.24.0
    \item \texttt{toml}: v0.10.2
    \item \texttt{XlsxWriter}: v3.2.0
    \item \texttt{matplotlib}: v3.9.2
    \item \texttt{openpyxl}: v3.1.5
    \item \texttt{ortools}: v9.11.4210
\end{itemize}
    \section{Appendix}

\begin{figure}[H]
    \centering
    \includegraphics[width=0.8\textwidth]{figures/Danske_Regioner__Sundhedsvæsenets_digitaliseringsrejse.png}
    \caption{The Danish healthcare sector's road to digitalisation}
    \medskip
    \small
    \raggedright
    The Danish healthcare sector's road to digitalisation, as presented by "Danske Regioner"\cite{Den-Reg-digitalisation}.
    \label{fig:healthcare_digitalisation}
\end{figure}

\begin{figure}[H]
    \centering
    \includegraphics[width=0.75\textwidth]{figures/Questionnaire_for_healthcare_sector.png}
    \caption{Questionnaire for healthcare personnel}
    \medskip
    \small
    \raggedright
    Questionnaire sent to Danish hospitals, in the pursuit of reaching out to healthcare personnel.
    \\
    \textbf{Translated from Danish:}
    \\
    \textbf{Title}: "Engineering project in the healthcare sector"
    \\
    \textbf{Introduction}: "Hello, my name is Felix, I am studying to become an engineer with a focus on Artificial intelligence at DTU. At the moment I am looking for a concrete problem for my master's thesis, and I was hoping to be able to work on a project, which combines technology with healthcare, in an attempt to make a real difference. In that pursuit, I wanted to ask whether you can spare a momemnt to answer this questionnaire." 
    \\
    \textbf{Question 1}: "What is the biggest and most frequent problem in your daily life as a healthcare professional?"
    \\
    \textbf{Question 2}: "What do you think is the biggest and most frequent problem in the daily life of your patients?"
    \\
    \textbf{Question 3}: "Please write your email and name, if I may contact you with follow-up questions."
    \\
    \textbf{Question 4}: "Name (not mandatory)"
    \\
    \textbf{Question 5}: "Email (not mandatory)"
    \label{fig:questionnaire_healthcare_personnel}
\end{figure}

    \printbibliography
\end{document}