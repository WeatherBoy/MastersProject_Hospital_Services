\documentclass{article}
\usepackage[english]{babel}
\usepackage[top=2cm, bottom=3cm, left=2cm, right=2cm]{geometry}

% ## Inserting Code ##
\usepackage{minted}
\usemintedstyle{fruity}

% ## Nice referencing ##
\usepackage{csquotes}   % <-- for biblatex
\usepackage{biblatex}
\addbibresource{Bibliography.bib}

% ## Math ##
\usepackage{amssymb}
\usepackage{amsmath}
\usepackage{amsthm}
\usepackage{cancel} % <-- the big slash over equations
\usepackage{breqn}  % <-- Automatic wrapping of long equations (good stuff)

% ## For the graphical aspect ##
\usepackage{float}  % <-- superior package for figure handling
\usepackage[dvipsnames]{xcolor}
\usepackage{graphicx} % <-- Required for inserting images
\usepackage{wrapfig} % <-- In case you want to wrap text around figures (can look quite pro)
\usepackage{arydshln}   % <-- For horizontal, vertical, etc. lines (for tables and matrices)
\usepackage{subcaption} % <-- For subfigures
\usepackage{mdframed} % <-- For frames around theorems and definitions

% ## Referencing ##
\usepackage{fancyref}   % <-- Supposed to make referencing better, don't know where it's used
\usepackage{hyperref}   % <-- has \autoref{} and all kinds of other goodies
\usepackage{url}   % <-- web addresses are handled properly when URL referencing

% ## Utility ##
\usepackage{enumitem}   % <-- enumerate, itemize and description
\usepackage{pdfpages}   % <-- for including external PDFs
\usepackage{titlesec}   % <-- (used for the header) section and chapter titles manipulation
\usepackage{subfiles}   % <-- smart for BIG projects where you include multiple subfiles
\usepackage{comment}    % <-- multi-line comments
\usepackage{fontawesome5}   % <-- add cool logos to your report
\usepackage{sectsty}    % <-- can help with generalising section design

\usepackage{fancyhdr}   % <-- for the header and footer
\pagestyle{fancy}   % <-- used to great extent for the title page header

% Make github logo
\usepackage{fontawesome5}
\newcommand{\github}[1]{%
   \href{#1}{\faGithubSquare}%
}

% For math definitions
\newmdtheoremenv{definition}{Definition}[section]
\newmdtheoremenv{theorem}{Theorem}[section]
\newmdtheoremenv{corollary}{Corollary}[section]

% ## For pseudocode ##
\usepackage{listings}
\lstset{
  basicstyle=\small\ttfamily,
  numbers=left,
  numberstyle=\tiny,
  stepnumber=1,
  frame=single,
  columns=flexible,
  keywordstyle=\bfseries\color{blue!70!black},
  commentstyle=\itshape\color{green!40!black},
  language=Python,
  showstringspaces=false
}

% Formalia goes here:
\newcommand\course{Master's Thesis} %<--- Kursus
\newcommand\datoen{F25} %<--- Dato
\newcommand\opgavetitel{The cumbersome necessity of digitalising the Danish healthcare sector} %<--- Navnet på opgaven


% Make GREAT looking header and footer:
\fancyhead[L]{Master's Thesis}
\fancyhead[R]{\rightmark}

\begin{document}
    \begin{titlepage}
    \begin{figure}[H]
         \centering
         \begin{subfigure}[H]{0.15 \textwidth}
             \centering
             \includegraphics[width=\textwidth]{figures/Titlepage/Coral_Red_RGB.pdf}
         \end{subfigure}
         \hfill
         \begin{subfigure}[H]{0.45\textwidth}
             \includegraphics[width=\textwidth]{figures/Titlepage/tex_dtu_compute_a_uk.pdf}
         \end{subfigure}
    \end{figure}
    \begin{center}
        \vspace{3 cm}
        \hrule
        \vspace{.3cm}
        { \huge {\bfseries {\opgavetitel}}
        } 
        % Title of the report:
        %The impact of regularization on classifiers and explaining the explainability of axiomatic attributions
        
        \vspace{.1cm}
        { \LARGE {\bfseries 
            {
                Master of Science in Engineering
                \\
                Human-Centered Artificial Intelligence
            }
        }
        }
        \vspace{.5cm}
        
        \hrule
        \vspace{1.5cm}
        
        \textbf{Authors}\\
        \vspace{.5cm}
        \centering
        
        % add your name here
        Felix Bo Caspersen | s183319\\
        
        \vspace{1.5cm}
        
        \textbf{Supervised by}\\
        \vspace{.5cm}
        \centering
        
        Professor Aasa Feragen \\
        \vspace{1.5cm}
        
        Kongens Lyngby \\
        \centering \datoen % Dags dato
    \end{center}
\end{titlepage}
    \begin{abstract}
    \textcolor{magenta}{TODO!}   
\end{abstract}
    \section*{Preface and Acknowledgements}
This project is written as part of the Master of Science in Human-Centered Artificial Intelligence at the Technical University of Denmark, Department of Applied Mathematics and Computer Science. Professor Aasa Feragen supervised this project. The project was completed from the 2nd of September 2024 to the 9th of March 2025. All the code supporting the project is in the GitHub repository \github{https://github.com/WeatherBoy/MastersProject_Hospital_Services}\footnote{For completeness: \href{https://github.com/WeatherBoy/MastersProject_Hospital_Services}{https://github.com/WeatherBoy/MastersProject\_Hospital\_Services}.}.
\\
\\
Please note that this report is the culmination of a five-year-long master's programme; as such, a fundamental mathematical and algorithmic foundation is expected of the reader.
\\
\\
I want to thank Aasa Feragen for her guidance in the project's overall direction. Furthermore, I owe a huge debt of gratitude to Annie Gam-Pedersen for not only acting as a liaison between me and OUH during the entire project but for without whom this project would never have been more than a questionnaire and a hope. Another huge thanks goes to the OUH staff, particularly Katrine and Rikke, for enduring my shenanigans. Additionally, I am deeply grateful for the clinical expertise of Anna Aakermann \O strup, which helped guide me through not only formulating the questionnaire but also interpreting the responses.
I would also like to thank Malthe \O rberg Gedsted and Tobias Drue for their tenacity in providing mission-critical feedback throughout the entire writing process.
Lastly, I would like to give my sincerest thanks to Tobias Drue, Rasmus Molsted and Jakob Kammeyer, for without whom and their unyielding support, I would almost assuredly have succumbed to despair.

    \setcounter{page}{1}
    \pagenumbering{roman}
    \tableofcontents
    \newpage

    \setcounter{page}{1}
    \pagenumbering{arabic}
    \section{Introduction}
The Danish healthcare sector is characterised by highly skilled medical professionals operating under constant and evergrowing time pressure.
Denmark's population is increasingly ageing, with a trend toward an older demographic. Statistics Denmark~\cite{dst-older-pop} states that the number of Danish citizens aged $80$ and above is expected to increase by $43$\% within the next decade, while the group aged $60$-$79$ is also expected to grow by $6.8$\%\footnote{The numerical increase is from $319,874$, in $2024$, to $457,488$, in $2034$, for $80$+ and $1,279,671$, in $2024$, to $1,359,430$, in $2034$ for $60$-$79$}. Moreover, recruiting qualified healthcare personnel is becoming increasingly difficult. The Ministry of Higher Education and Science~\cite{MoHEaS-healthcare-rectruitment} reports that, although there has been a recent increase in the number of students graduating in healthcare-related fields, this is still insufficient, and the lack of qualified healthcare personnel is only expected to worsen over time. The Ministry of Finance~\cite{MoHEaS-healthcare-rectruitment} anticipates that the healthcare sector will lack approximately $15,000$ SOSU workers by $2035$, and in response, the government plans to partner with developing countries\footnote{Mentioned are India and the Phillipines.} to help recruit healthcare personnel.
\\
\\
Over the past three decades\footnote{See \autoref{fig:healthcare_digitalisation}, 'Danske Regioner' made a pleasant visualisation.}, digital initiatives and platforms have been developed in an attempt to alleviate the challenges posed by an understaffed healthcare sector by enhancing workflow efficiency and patient care. However, due to the complexity within the healthcare sector arising from diverse patient needs, intricate protocols and procedures, and a plethora of administrative tasks, the resulting systems put in place\footnote{The classic example here being "Sundhedsplatformen", Epic's healthcare-IT system, that since its inception in 2016, has been under constant critique~\cite{DR-healthcare-platform, Altinget-healthcare-platform, TV2-healthcare-platform}. Though, in all honesty, in 2021 in a user survey by "Center for Patientinddragelse"~\cite{SP-user-survey} they noticed a slight uptick in general satisfaction amongs users i.e. the healthcare staff.} are often challenging for healthcare personnel to use effectively. While the initiative behind these solutions is commendable, the final products usually falls short. The solutions can be cumbersome, unintuitive, and prone to disrupt daily routines. In this context, the importance of human-centred design and usability cannot be overstated. Without careful attention to user experience, even the most technologically advanced solution risks going unused, ultimately failing to support the healthcare personnel it was designed to assist.
\\
\\
This thesis emerges from a desire to address the real-world, everyday challenges faced by healthcare personnel. Initially, the hope was to further explore modern techniques such as Large Language Models (LLMs) in order to streamline the documentation process. However, first-hand contact with the sector revealed higher priority problems. Rather than pursuing a complex, theoretically appealing artificial intelligence (AI) solution, the focus pivoted toward more fundamental digital tools. These were threefold:
\begin{enumerate}
    \item Digitalising a taskboard system, which currently requires manual data transcription by a nurse at the start of each day.
    \item Developing a scheduling assistance tool to help administrative staff assign personnel while adhering to various practical constraints.
    \item Automating the validation process needed to check the final schedule against management's desired plan.
\end{enumerate}
None of these tasks proved as straightforward as anticipated. The initial concept for the taskboard digitalisation suggested a relatively simple data-integration project. However, the implementation process has extended over several months due to unforeseen complexities in data formats, infrastructure limitations, evolving requirements and more.


    \section{Theory}
Of the three main tasks explored in this thesis, developing a \textit{scheduling assistance tool}, which we henceforth will refer to as '\textit{the scheduling task}', required the most mathematical approach. \textit{The scheduling task} is a case of the \textbf{Generalised Assignment Problem}\cite{Wiki-general-assignment-prob}, which can ultimately be expressed as a \textbf{Linear Program}\cite{Wiki-linear-programming}, specifically, a \textbf{Binary Integer Program}.
The following sections explore different theoretical approaches that might yield a satisfactory solution to such a \textbf{Generalised Assignment Problem}.

\subsection{Genetic Algorithms}
A Genetic Algorithm (GA), as popularised by John Holland\cite{Genetic-Algorithm-original} in 1975\footnote{Though later revised in 1992.}, is a metaheuristic within primarily \textbf{Computer Science} and \textbf{Operations Research} (OR). Inspired by evolutionary theory's natural selection, a GA is typically employed for \textbf{Optimisation} and \textbf{Search problems}.
\\
In short, inspired by natural selection, a GA scheme initiates, at random, a pool of \textit{candidate solutions} called the \textit{initial generation}. It then evaluates the solutions with respect to the given domain and assigns each of them a \textit{fitness score}. Finally, utilising some selection criteria, the GA scheme pairs the solutions with the best fitness, creating a new set of candidate solutions called the \textit{next generation}. However, in order to avoid converging towards some \textit{local optimum}, a bit of stochasticity (\textit{mutation}) is introduced in each generation. The GA scheme then iterates this entire process, stopping when reaching some criteria, either time or a threshold on the fitness score.
\\
\\
A Genetic Algorithm, boiled down to its base parts, consists of an objective function $f : \mathbb{R}^N \rightarrow [0,1]$, for some $N \in \mathbb{N}$, a \textit{selection criteria}, and a population of \textit{phenotypes}, the candidate solutions - each with its own genotype. A population of phenotypes is formally called the $i$'th generation, where $i$ adheres to the current iteration of the GA scheme. The number of phenotypes in a generation is usually chosen as a power of two, $2^N$, as this can significantly impact computer performance. The phenotypes and genotypes are domain-dependent; however, they are frequently represented as N-dimensional vectors. The genotypes will often be the actions or set of rules from which the phenotype can be derived. However, this is also largely dependent on the domain. For some problems, the genotype and phenotype are simply identical.
\\
Finally, the objective function $f$ and selection criteria are largely the crux of the GA scheme. While the phenotypes and their respective genotypes are essentially bound to the domain, the objective function and selection criteria can be modelled for better results. The objective function and selection criteria could be considered the Genetic Algorithm's counterpart to Machine Learning's (ML) hyperparameters.
\\
The objective function's sole criteria is that it evaluates to a scalar, usually standardised to the real interval $[0,1]$, hence $f$ doesn't need to be differentiable nor even continuous. This feature can allow for some creative objective functions specifically suited to the given problem.
\\
The selection criteria is how we pick the \textit{parents} for the next generations. The parents, a set of two phenotypes, always produce two \textit{children} to keep the population size constant throughout the iterations of the GA scheme. However, picking the parents can be a science in itself. Detailed below are some different algorithms used to perform the selection step.

\subsubsection{Random Selection}
Like its name gives way to, Random Selection simply chooses a parent uniformly at random from the entire set of $M$ phenotypes. A big plus for Random Selection is that it finds a candidate in $O(1)$ time without preprocessing. 

\subsubsection{Fitness Proportionate Selection}
Fitness Proportionate Selection, or as it is more often and very aptly called Roulette Wheel Selection, selects a phenotype weighted by its fitness. Mathematically, we can express it as the probability $p_i$ of selecting the $i$th phenotype being
\begin{equation*}
    p_i = \frac{f_i}{\sum_{j = 1}^M f_j},
\end{equation*}
where $f_i$ is the fitness score of the $i$th phenotype.
\\
For the implementation, one would typically normalise the fitness scores and then create $M$ bins of cumulative ranges, each corresponding to their respective fitness scores. Then, "spinning the boule" would equate to generating a \textit{floating point} value in the range $[0,1]$ uniformly at random and performing a binary search for finding the corresponding bin. This implementation would take $O(M)$ preprocessing and $O(\log M )$ at runtime. 

\begin{figure}[H]
    \centering
    \begin{subfigure}[b]{0.8\textwidth}
        \centering
        \includegraphics[width=\textwidth]{figures/Genetic_Algorithms/Roulette_Selection.png}
        \subcaption{Five phenotypes (A, B, C, D and E) with fitness scores $0.1$, $0.2$, $0.05$, $0.3$, $0.4$.}
    \end{subfigure}
    \begin{subfigure}[b]{0.8\textwidth}
        \centering
        \includegraphics[width=\textwidth]{figures/Genetic_Algorithms/Roulette_Selection2.png}
        \subcaption{Five phenotypes (A, B, C, D and E) with fitness scores $0.1$, $0.2$, $0.05$, $0.3$, $0.4$. Scaled to their fitness, with a boule for the sake of example.}
    \end{subfigure}
    
    \caption{Roulette Wheel Selection - Visualisation}
    \small
    \raggedright
    An attempt at visualising the idea behind Roulette Wheel Selection. First the five phenotypes, with their given fitness scores, then below, scaled to show the increased probability of the boule stopping on a phenotype with a higher fitness score. (Despite the roulette wheel here being mapped to 1D, the example still stands. As it prooved far too difficult to draw a 2D roulette wheel, we must rely on the reader's imagination.)
    \label{fig:GA_Roulette_Selection}
\end{figure}

\subsubsection{Tournament Selection}
Tournament Selection selects a subset of size\footnote{As with $M$, it is preferable to pick $k$ as a power of two.} $1 \leq k \leq M$, uniformly at random of the generation and elects the phenotype with the greatest fitness score, the \textit{best candidate}, within the subset as the winner or, in other words, a parent of the next generation. One might note, that choosing $k = 1$ is equivalent to Random Selection.
\\
A variant\cite{Wiki-tournament-selection} of Tournament Selection introduces a probability $p$ of electing the best candidate as the winner. Inversely, with probability $1-p$, a new tournament is held, where the best candidate is removed, leaving $k-1$ phenotypes, where once again, the best candidate of the remaining phenotypes is selected with probability $p$, and with probability $1-p$, we repeat the process. This variant can, in turn, be considered a Geometric Distribution, with each tournament being a Bernoulli Trial with probability of success $p$. Due to the properties\cite{Wiki-geometric-distribution} of a Geometric Distribution, we can infer that the probability of picking the $k$th best candidate is
\begin{equation*}
    P\left[ X = k \right] = (1-p)^{k-1} \cdot p.
\end{equation*}

Recalling that only a single pair of children (next generation) can be derived from a pair of parents (current generation), for a generation size of $M$, the GA scheme requires exactly $M$ parents to produce the next generation. Since Tournament Selection only produces one parent per tournament, it would require $M$ tournaments between every generation.

    \section{Methodology}
\subsection{First Steps - The Questionnaire}
Rather than researching some obscure nook or cranny of mathematics, this project was initially motivated by a desire to try and implement 5 years of education in order to assist with some current real-world problems. From prior work, we hypothesised that there might be quite a handful of such problems within the Danish healthcare sector. As such, we laid the groundwork in March of 2024 by sending out a questionnaire, which can be seen in appendix \autoref{fig:questionnaire_healthcare_personnel}, to \textbf{Rigshospitalet}, \textbf{The Hospital of Hvidovre}, \textbf{The regional Hospital of Northern Jutland}, \textbf{The University Hospital of Aalborg}, \textbf{Hospital unit Midt}, \textbf{The Hospital of Holbæk} and \textbf{The department of orthopaedic surgery at the University Hospital of Odense}. This was met with quite varying responses. Most deemed that answering such a questionnaire would be an abhorrent misuse of their personnel's limited time. This, admittedly, left us a bit discouraged. However, we still received a total of 39 answers from medical professionals.
\\
The questionnaire consisted of two questions\footnote{With the option to add your name and/or mail for possible further inquiry.}:
\begin{itemize}
    \item What is the biggest and most frequent problem in your daily life as a healthcare professional?
    \item What do you think is the biggest and most frequent problem in the daily life of your patients?
\end{itemize}
The responses about the daily life and tasks of healthcare professionals varied significantly, but they could likely be summarised as a general dissatisfaction with the understaffed and underfunded state of the healthcare sector.
\\
Many complained about the intricate journaling processes - documenting the patients' treatment pathways, reading through a lot of unnecessary documentation and documenting the same thing twice. 
\\
There was a plethora of replies criticising the IT systems currently in place. Complaining that they either didn't work, were too slow or kept crashing. Additionally, one respondee noted that they have a lot of programs that advertise useful functionality, but because they aren't integrated with all the other individual solutions, the healthcare staff end up having to open multiple programs documenting the same data many times over.
\\
Some mentioned that there are a lot of wrongful queries to the emergency line that have to be redirected. It was also added that a filter or automatic redirection might save a lot of critical time.
\\
Finally, the vast majority of responses were aimed at a lack of resources. This included being understaffed, resulting in insufficient time to care for patients and individual staff members being assigned too many tasks. Additionally, there was a shortage of critical inventory, such as bed space and essential medical equipment. Moreover, the existing equipment was often old or outdated due to budget cuts.
\\
\\
The responses regarding the everyday problems of patients weren't nearly as varied. These were mainly regarding long waiting times and a lack of transparency. Out of 39 responses, 12 mentioned that patients had expressed concerns about unclear treatment pathways. Additionally, 18 out of 39 responses indicated that patients had complained about the length of the waiting times. Lastly, a handful of respondents (approximately 6) mentioned that their patients feel overlooked, which may also indicate an understaffed healthcare sector.
\\
\\
After reading the responses to our questionnaire and working in tandem with the healthcare sector for roughly six months, it has become undoubtedly apparent that the healthcare sector is in dire need of a digital revolution. A plethora of tasks could be expedited seamlessly without contributing additional burdens to the medical personnel if only proper digital solutions were implemented. However, we would have been unable to work on any practical application within the healthcare sector if all responses we had received were equally uninclined to collaborate. Therefore, we were overjoyed when the University Hospital of Odense (OUH) reached out with an incredibly accommodating response.

\subsection{The Digital Taskboard}
OUH responded that their department of Orthopaedic surgery uses a whiteboard as a daily taskboard. A picture of the current whiteboard is shown in \autoref{fig:Whiteboard-Taskboard}. However, this whiteboard is being drawn up daily by manually transcribing data from two distinct spreadsheets. OUH  informed us that the process usually takes 10-20 minutes, time that could be better spent on more critical tasks by the clinical staff, and as such, they wished for a digital replacement for the whiteboard. Therefore, this thesis began with the seemingly straightforward task of merging two 2D datasets and visualising them on a screen comparable to a whiteboard.
\\
For clarity, we would like to note that the two spreadsheets are separated purely for administrative purposes; the separation doesn't add any additional dimensionality. Hence, it can still be displayed without loss of dimensionality on a plane like a whiteboard or screen. 

\begin{figure}[H]
    \centering
    \includegraphics[width=0.9\textwidth]{figures/Methodology/Whiteboard-Taskboard-Redacted.png}
    \caption{Whiteboard Taskboard}
    \small
    \raggedright 
    A picture of the taskboard currently in place at OUH. Names and tasks have been redacted to preserve the anonymity of OUH's personnel. 
    \label{fig:Whiteboard-Taskboard}
\end{figure}
    \section{Analysis}
\textit{The scheduling task} is a case of the \textbf{Generalised Assignment Problem}\cite{Wiki-general-assignment-prob}, which can ultimately be expressed as a \textbf{Linear Program}\cite{Wiki-linear-programming}, specifically, a \textbf{Binary Integer Program}.

However, in order to arrive at CP, we will first argue, without a complete theoretical deep dive, that the Generalised Assignment Problem can be reformulated into the binary case of an integer linear problem (ILP). A binary ILP takes the form\cite{Integer-Programming-Book} 
\begin{align*}
    \text{max} \quad &\sum_{i = 1}^m \sum_{j = 1}^n c_{ij} \cdot x_{ij}
    \\
    \text{subject to} \quad &\sum_{j = 1}^n a_{ij} \cdot x_{ij} \leq b \qquad &\text{for} \quad i = 1, \hdots, m
    \\
    &\sum_{i = 1}^m x_{ij} \leq 1 \qquad &\text{for} \quad j = 1, \hdots, n
    \\
    \text{for} \quad &x \in \{0, 1\}^{m \times n},
\end{align*}
where $x$ is the binary decision variable.
\\
This formulation gives way to a Boolean Satisfiability Problem (BSAT), a subfield of CP, which we can attempt to solve with a \textit{satisfiability} (SAT) solver.
\\
\\
As a last remark on the technical implementation aspect, we will quickly address the running time. As exemplified in \autoref{eq:decision-variable-form}, we will have a decision variable for each agent \(a \in [0, M]\), for each day \(i \in [0, T]\), for each task \(j \in [0, N]\). This is a nice representation, but a better and more straightforward definition is that we have a sum of \(N\) tasks over all \(T\) days. That way, we can readily define the total number of decision variables in our boolean formula as \(M \cdot N\) without having to deal with having an unequal amount of tasks for each respective day. As we have stated many times, boolean satisfiability is NP-complete. Therefore, we ultimately end up with a running time of \(2^{M \cdot N}\).
    \section{Discussion}
We started this project by agreeing to provide OUH with a digital replacement for their current, manually transcribed taskboard (as detailed in \autoref{sec:Digital-Taskboard}). While true that this problem, on its own, couldn't provide a sufficient academic foundation for a Master's thesis in AI, we nevertheless made the grave error of underestimating just how cumbersome\footnote{Which, to her great credit, our supervisor had actually warned us about.} of a task digitalising the taskboard could be.
\\
\\
At a glance, digitalising the taskboard seemed no more complex than merging two 2D datasets, and we are still convinced it could have been were it not for OUH's currently in-place payroll and calendar overview system, HosInfo, serving as an ever-persistent thorn in our side. While the HosInfo implementation could be criticised plentily for its sluggish and featureless backend\footnote{Which we admittedly never had access to but only have seen in parsing when been guided through the system by the OUH staff.}, beyond the shadow of a doubt, our biggest gripe with it was its data handling or, rather, lack thereof. Not only does HosInfo make OUH's data nearly inaccessible by storing text data as pictures, but they also enforce no data formatting on their users. Additionally, rather than designing a solution capable of serving the greater planning effort at OUH, they have delivered a solution to a smaller subproblem while completely disregarding even the slightest notion of compatibility with the rest of OUH's IT environment.
\\
Such software makes it nigh impossible for OUH, not even to mention other external providers such as ourselves, to develop other solutions that need access to their data without forcing the end user to document everything many times over.
\\ 
However, HosInfo isn't the only third-party provider that is guilty of contributing to OUH's messy, tangled, and, at the same time, disconnected IT stack. We can only speculate on the reasons behind how it ends up as such. It might be due to the Danish and European procurement legislation~\cite{Udbudsloven, EU-Procurement-Legislation}, or in part due to a lack of technical insight at OUH, which limits them in seeing the greater scope of their administrative challenges. Nevertheless, it culminates in a system comprising a slew of individual IT solutions, which can't share data nor benefit from what each provides.
\\
Trying to work around this constant impediment is perfectly summarised by this single response to our questionnaire, which loosely translated into English would be:
\begin{center}
 \textit{We have many systems that can do quite a lot, but not always in conjunction with each other, and most of the time, you have to work across several applications.}
\end{center}
While (our interpretation of) this statement perfectly encapsulates our immense frustration after six months of efforts of working with (or around) the Danish healthcare sector's IT, we can still scarcely imagine having to work with it on a daily basis. Therefore, we thoroughly believe that a more cohesive and all-encompassing platform would be the right move.
\\
We envision a cloud-based platform with data formatting requirements which can serve data to the correct users. Such a platform should have a single standardised user interface (UI) and allow the administrative workers to extract and log data for their corresponding hospitals (or other clinical facilities). It should facilitate an API allowing approved independent third-party providers to create individual software solutions that could work off of this platform. Such a platform would immediately eliminate any need to log data multiple times, as it would all be gathered in one place. Furthermore, if such a platform enforced a UI design guide onto third-party providers and required them to implement their solution onto the platform, it would save the healthcare personnel from having multiple credentials and remembering different layouts for each and every single individual application.
\\
This idea would require a lot from the IT sectors of the Danish healthcare sector; in turn, we also hope that it would be easier to maintain if all of it was gathered onto a single platform, with the single uniform goal of facilitating more seamless implementation of IT solutions.
\\
However, this isn't exactly 'new tech', as "Danske Regioner" in 2024 proposed a 'Danish Healthcare Cloud'~\cite{Den-Reg-digitalisation}. The referenced note touches upon many of the same topics as we have just detailed. Rather than feeling amiss for only arriving at the same conclusion as somebody else had found way before us, we will instead let it enforce our belief that such a restructuring wouldn't simply be an improvement; it is a requirement.
\\
\\
This brings us nicely along to our next point, for although, in all likelihood, we wouldn't be qualified to implement such a platform, we did attempt to build The Scheduling Assistant tool, which we believe could have been a good starting point for OUH, at least.
\\
As mentioned, we didn't believe digitalising the OUH taskboard would be academically sufficient; therefore, we also pursued the scheduling task. We believed that we could implement this side-by-side with the digital taskboard, and while we still think it technically feasible, as it has been done before~\cite{you-but-better} (and to a way greater extent), much like for the digital taskboard, we ran into multiple unforeseen difficulties.
    \section{Conclusion}

    \listoffigures
    \newpage

    \listoftables
    \newpage

    \section{Software versions used}
\begin{itemize}
    \item \texttt{beautifulsoup4}: v4.12.3
    \item \texttt{pandas}: v2.2.2
    \item \texttt{selenium}: v4.24.0
    \item \texttt{toml}: v0.10.2
    \item \texttt{XlsxWriter}: v3.2.0
    \item \texttt{matplotlib}: v3.9.2
    \item \texttt{openpyxl}: v3.1.5
    \item \texttt{ortools}: v9.11.4210
\end{itemize}
    \section{Appendix}

\begin{figure}[H]
    \centering
    \includegraphics[width=0.8\textwidth]{figures/Danske_Regioner__Sundhedsvæsenets_digitaliseringsrejse.png}
    \caption{The Danish healthcare sector's road to digitalisation}
    \medskip
    \small
    \raggedright
    The Danish healthcare sector's road to digitalisation, as presented by "Danske Regioner"\cite{Den-Reg-digitalisation}.
    \label{fig:healthcare_digitalisation}
\end{figure}

\begin{figure}[H]
    \centering
    \includegraphics[width=0.75\textwidth]{figures/Questionnaire_for_healthcare_sector.png}
    \caption{Questionnaire for healthcare personnel}
    \medskip
    \small
    \raggedright
    Questionnaire sent to Danish hospitals, in the pursuit of reaching out to healthcare personnel.
    \\
    \textbf{Translated from Danish:}
    \\
    \textbf{Title}: "Engineering project in the healthcare sector"
    \\
    \textbf{Introduction}: "Hello, my name is Felix, I am studying to become an engineer with a focus on Artificial intelligence at DTU. At the moment I am looking for a concrete problem for my master's thesis, and I was hoping to be able to work on a project, which combines technology with healthcare, in an attempt to make a real difference. In that pursuit, I wanted to ask whether you can spare a momemnt to answer this questionnaire." 
    \\
    \textbf{Question 1}: "What is the biggest and most frequent problem in your daily life as a healthcare professional?"
    \\
    \textbf{Question 2}: "What do you think is the biggest and most frequent problem in the daily life of your patients?"
    \\
    \textbf{Question 3}: "Please write your email and name, if I may contact you with follow-up questions."
    \\
    \textbf{Question 4}: "Name (not mandatory)"
    \\
    \textbf{Question 5}: "Email (not mandatory)"
    \label{fig:questionnaire_healthcare_personnel}
\end{figure}

    \printbibliography
\end{document}