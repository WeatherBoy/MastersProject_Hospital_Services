\begin{abstract} 
    Denmark's healthcare sector faces mounting pressure from an ageing population, staff shortages, and expanding administrative demands. In response, this thesis explores how targeted digital tools can alleviate some of the everyday burdens placed on clinical personnel. First, we investigate the process of digitalising a daily whiteboard - used to coordinate tasks - by merging data from multiple, often incompatible sources. Here, the main challenge was to parse and reconcile differing formats, exposing deeper systemic problems in data integration.
    \\
    Next, we propose a scheduling assistant tool employing constraint programming to automate the allocation of staff across tasks, subject to intricate real-world rules. While our prototypes highlight the feasibility and potential efficiency gains of such an approach, they also underscore the need for robust data handling and a user-friendly interface that bridges informal requirements and formal logic. Finally, we outline an automated schedule validator intended to verify whether final schedules remain faithful to overarching managerial plans.
    \\
    Collectively, these contributions illustrate the promise of digital solutions in streamlining clinical workflows and the practical difficulties - ranging from entrenched practices to isolated IT systems - that hamper meaningful progress. Despite the partial nature of our prototypes, our work indicates that combining methodical constraint modelling with human-centred design can yield functioning strategies for modernising administrative routines in the Danish healthcare sector.
\end{abstract}