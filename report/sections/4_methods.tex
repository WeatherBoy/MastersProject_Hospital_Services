\section{Methodology}
\subsection{First Steps - The Questionnaire}
Rather than researching some obscure nook or cranny of mathematics, this project was initially motivated by a desire to try and implement 5 years of education in order to assist with some current real-world problems. From prior work, we hypothesised that there might be quite a handful of such problems within the Danish healthcare sector. As such, we laid the groundwork in March of 2024 by sending out a questionnaire, which can be seen in appendix \autoref{fig:questionnaire_healthcare_personnel}, to \textbf{Rigshospitalet}, \textbf{The Hospital of Hvidovre}, \textbf{The regional Hospital of Northern Jutland}, \textbf{The University Hospital of Aalborg}, \textbf{Hospital unit Midt}, \textbf{The Hospital of Holbæk} and \textbf{The department of orthopaedic surgery at the University Hospital of Odense}. This was met with quite varying responses. Most deemed that answering such a questionnaire would be an abhorrent misuse of their personnel's limited time. This, admittedly, left us a bit discouraged. However, we still received a total of 39 answers from medical professionals.
\\
The questionnaire consisted of two questions\footnote{With the option to add your name and/or mail for possible further inquiry.}:
\begin{itemize}
    \item What is the biggest and most frequent problem in your daily life as a healthcare professional?
    \item What do you think is the biggest and most frequent problem in the daily life of your patients?
\end{itemize}
The responses about the daily life and tasks of healthcare professionals varied significantly, but they could likely be summarised as a general dissatisfaction with the understaffed and underfunded state of the healthcare sector.
\\
Many complained about the intricate journaling processes - documenting the patients' treatment pathways, reading through a lot of unnecessary documentation and documenting the same thing twice. 
\\
There was a plethora of replies criticising the IT systems currently in place. Complaining that they either didn't work, were too slow or kept crashing. Additionally, one respondee noted that they have a lot of programs that advertise useful functionality, but because they aren't integrated with all the other individual solutions, the healthcare staff end up having to open multiple programs documenting the same data many times over.
\\
Some mentioned that there are a lot of wrongful queries to the emergency line that have to be redirected. It was also added that a filter or automatic redirection might save a lot of critical time.
\\
Finally, the vast majority of responses were aimed at a lack of resources. This included being understaffed, resulting in insufficient time to care for patients and individual staff members being assigned too many tasks. Additionally, there was a shortage of critical inventory, such as bed space and essential medical equipment. Moreover, the existing equipment was often old or outdated due to budget cuts.
\\
\\
The responses regarding the everyday problems of patients weren't nearly as varied. These were mainly regarding long waiting times and a lack of transparency. Out of 39 responses, 12 mentioned that patients had expressed concerns about unclear treatment pathways. Additionally, 18 out of 39 responses indicated that patients had complained about the length of the waiting times. Lastly, a handful of respondents (approximately 6) mentioned that their patients feel overlooked, which may also indicate an understaffed healthcare sector.
\\
\\
After reading the responses to our questionnaire and working in tandem with the healthcare sector for roughly six months, it has become undoubtedly apparent that the healthcare sector is in dire need of a digital revolution. A plethora of tasks could be expedited seamlessly without contributing additional burdens to the medical personnel if only proper digital solutions were implemented. However, we would have been unable to work on any practical application within the healthcare sector if all responses we had received were equally uninclined to collaborate. Therefore, we were overjoyed when the University Hospital of Odense (OUH) reached out with an incredibly accommodating response.

\subsection{The Digital Taskboard}
OUH responded that their department of Orthopaedic surgery currently uses a whiteboard as a daily taskboard. However, this whiteboard is being drawn up daily by manually transcribing data from two distinct spreadsheets. OUH  informed us that the process usually takes 10-20 minutes, time that could be better spent on more critical tasks by the clinical staff, and as such, they wished for a digital replacement for the whiteboard. Therefore, this thesis began with the seemingly straightforward task of merging two 2D datasets and visualising them on a screen comparable to a whiteboard.
\\
For clarity, we would like to note that the two spreadsheets are separated purely for administrative purposes; the separation doesn't add any additional dimensionality. Hence, it can still be displayed on a plane surface like a whiteboard or screen. 

\begin{figure}[H]
    \centering
    \includegraphics[width=0.9\textwidth]{figures/Methodology/Whiteboard-Taskboard-Redacted.png}
    \caption{Whiteboard Taskboard}
    \small
    \raggedright 
    A picture of the taskboard currently in place at OUH. Names and tasks have been redacted to preserve the anonymity of OUH's personnel. 
    \label{fig:Whiteboard-Taskboard}
\end{figure}