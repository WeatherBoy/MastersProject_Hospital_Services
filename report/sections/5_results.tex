\section{Results}
\textit{The scheduling task} is a case of the \textbf{Generalised Assignment Problem}\cite{Wiki-general-assignment-prob}, which can ultimately be expressed as a \textbf{Linear Program}\cite{Wiki-linear-programming}, specifically, a \textbf{Binary Integer Program}.

However, in order to arrive at CP, we will first argue, without a complete theoretical deep dive, that the Generalised Assignment Problem can be reformulated into the binary case of an integer linear problem (ILP). A binary ILP takes the form\cite{Integer-Programming-Book} 
\begin{align*}
    \text{max} \quad &\sum_{i = 1}^m \sum_{j = 1}^n c_{ij} \cdot x_{ij}
    \\
    \text{subject to} \quad &\sum_{j = 1}^n a_{ij} \cdot x_{ij} \leq b \qquad &\text{for} \quad i = 1, \hdots, m
    \\
    &\sum_{i = 1}^m x_{ij} \leq 1 \qquad &\text{for} \quad j = 1, \hdots, n
    \\
    \text{for} \quad &x \in \{0, 1\}^{m \times n},
\end{align*}
where $x$ is the binary decision variable.
\\
This formulation gives way to a Boolean Satisfiability Problem (BSAT), a subfield of CP, which we can attempt to solve with a \textit{satisfiability} (SAT) solver.
\\
\\
As a last remark on the technical implementation aspect, we will quickly address the running time. As exemplified in \autoref{eq:decision-variable-form}, we will have a decision variable for each agent \(a \in [0, M]\), for each day \(i \in [0, T]\), for each task \(j \in [0, N]\). This is a nice representation, but a better and more straightforward definition is that we have a sum of \(N\) tasks over all \(T\) days. That way, we can readily define the total number of decision variables in our boolean formula as \(M \cdot N\) without having to deal with having an unequal amount of tasks for each respective day. As we have stated many times, boolean satisfiability is NP-complete. Therefore, we ultimately end up with a running time of \(2^{M \cdot N}\).