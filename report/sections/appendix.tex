\section{Appendix}
\subsection{Examples}
\subsubsection{Propositional logic to Conjunctive Normal Form}
In the following, we will go through the entire procedure of converting this formula
\begin{equation}\label{eq:CNF_example1}
    \Big(
        p_1 \leftrightarrow (
            p_2 \vee \neg p_3
        )
    \Big) \wedge \Big(
        p_5 \rightarrow (
            p_1 \uparrow p_4
        )
    \Big) \wedge \Big(
        p_4 \oplus p_2
    \Big),
\end{equation}
 to CNF.
\\
For easy reference, we will redeclare the logical equivalence rules from \autoref{sec:CNF}.
\begin{equation}\label{eq:biimplication}
    A \leftrightarrow B \equiv (A \rightarrow  B) \wedge (B \rightarrow A)
\end{equation}
\begin{equation}\label{eq:XOR}
    A \oplus B \equiv \neg (A \rightarrow B) \vee \neg (B \rightarrow A)
\end{equation}
\begin{equation}\label{eq:implication}
    A \rightarrow B \equiv \neg A \vee B
\end{equation}
\begin{equation}\label{eq:NAND}
    A \uparrow B \equiv \neg (A \wedge B)
\end{equation}
\begin{equation}\label{NOR}
    A \downarrow B \equiv \neg (A \vee B)
\end{equation}
Taking it step-by-step, first, we eliminate all boolean operators except for $\neg$, $\vee$ and $\wedge$.
\\
Using \autoref{eq:biimplication} on \autoref{eq:CNF_example1}:
\begin{equation}\label{eq:CNF_example2}
    \bigg(
        \Big(
            p_1 \rightarrow (
                p_2 \vee \neg p_3
            )
        \Big) \wedge \Big(
                (
                    p_2 \vee \neg p_3 
                ) \rightarrow p_1 
            \Big)
    \bigg) \wedge \Big(
        p_5 \rightarrow (
            p_1 \uparrow p_4
        )
    \Big) \wedge \Big(
        p_4 \oplus p_2
    \Big).
\end{equation}
Using \autoref{eq:implication} on \autoref{eq:CNF_example2}:
\begin{equation}\label{eq:CNF_example3}
    \bigg(
        \Big(
            \neg p_1 \vee (
                p_2 \vee \neg p_3
            )
        \Big) \wedge \Big(
                \neg (
                    p_2 \vee \neg p_3 
                ) \vee p_1 
            \Big)
    \bigg) \wedge \Big(
        \neg p_5 \vee (
            p_1 \uparrow p_4
        )
    \Big) \wedge \Big(
        p_4 \oplus p_2
    \Big).
\end{equation}
Using \autoref{eq:NAND} on on \autoref{eq:CNF_example3}:
\begin{equation}\label{eq:CNF_example4}
    \bigg(
        \Big(
            \neg p_1 \vee (
                p_2 \vee \neg p_3
            )
        \Big) \wedge \Big(
                \neg (
                    p_2 \vee \neg p_3 
                ) \vee p_1 
            \Big)
    \bigg) \wedge \Big(
        \neg p_5 \vee \neg (
            p_1 \wedge p_4
        )
    \Big) \wedge \Big(
        p_4 \oplus p_2
    \Big).
\end{equation}
Using \autoref{eq:XOR} on on \autoref{eq:CNF_example4}:
\begin{equation}\label{eq:CNF_example5}
    \bigg(
        \Big(
            \neg p_1 \vee (
                p_2 \vee \neg p_3
            )
        \Big) \wedge \Big(
                \neg (
                    p_2 \vee \neg p_3 
                ) \vee p_1 
            \Big)
    \bigg) \wedge \Big(
        \neg p_5 \vee \neg (
            p_1 \wedge p_4
        )
    \Big) \wedge \Big(
        \neg (
            p_4 \rightarrow p_2
        ) \vee \neg (
            p_2 \rightarrow p_4
        )
    \Big).
\end{equation}
Using \autoref{eq:implication} on on \autoref{eq:CNF_example5}:
\begin{equation}\label{eq:CNF_example6}
    \bigg(
        \Big(
            \neg p_1 \vee (
                p_2 \vee \neg p_3
            )
        \Big) \wedge \Big(
                \neg (
                    p_2 \vee \neg p_3 
                ) \vee p_1 
            \Big)
    \bigg) \wedge \Big(
        \neg p_5 \vee \neg (
            p_1 \wedge p_4
        )
    \Big) \wedge \Big(
        \neg (
            \neg p_4 \vee p_2
        ) \vee \neg (
            \neg p_2 \vee p_4
        )
    \Big).
\end{equation}
Second, we propagate negations inward:
\begin{equation*}
    \bigg(
        \Big(
            \neg p_1 \vee (
                p_2 \vee \neg p_3
            )
        \Big) \wedge \Big(
                (
                    \neg p_2 \wedge \neg \neg p_3 
                ) \vee p_1 
            \Big)
    \bigg) \wedge \Big(
        \neg p_5 \vee (
            \neg p_1 \vee \neg p_4
        )
    \Big) \wedge \Big(
        (
            \neg \neg p_4 \wedge \neg p_2
        ) \vee (
            \neg \neg p_2 \wedge \neg p_4
        )
    \Big).
\end{equation*}
Third, we remove any redundant double negations:
\begin{equation}\label{eq:CNF_example7}
    \bigg(
        \Big(
            \neg p_1 \vee (
                p_2 \vee \neg p_3
            )
        \Big) \wedge \Big(
            (
                \neg p_2 \wedge p_3 
            ) \vee p_1 
            \Big)
    \bigg) \wedge \Big(
        \neg p_5 \vee (
            \neg p_1 \vee \neg p_4
        )
    \Big) \wedge \Big(
        (
            p_4 \wedge \neg p_2
        ) \vee (
            p_2 \wedge \neg p_4
        )
    \Big).
\end{equation}
Next, we put \autoref{eq:CNF_example7} in CNF by utilising the distributive property of boolean operators:
\begin{gather*}
    \bigg(
        \Big(
            \neg p_1 \vee (
                p_2 \vee \neg p_3
            )
        \Big) \wedge \Big(
            (
                p_1 \vee \neg p_2 
            ) \wedge (
                p_1 \vee p_3
            ) 
            \Big)
    \bigg) \wedge \Big(
        \neg p_5 \vee (
            \neg p_1 \vee \neg p_4
        )
    \Big) \wedge \bigg(
        \Big(
            (
                p_4 \wedge \neg p_2
            ) \vee p_2
        \Big) \vee \Big(
            (
                p_4 \wedge \neg p_2
            ) \vee \neg p_4 
        \Big)
    \bigg)
    \\
    \equiv
    \\
    \bigg(
        \Big(
            \neg p_1 \vee (
                p_2 \vee \neg p_3
            )
        \Big) \wedge \Big(
            (
                p_1 \vee \neg p_2 
            ) \wedge (
                p_1 \vee p_3
            ) 
            \Big)
    \bigg) \wedge \Big(
        \neg p_5 \vee (
            \neg p_1 \vee \neg p_4
        )
    \Big) \wedge
    \\
    \bigg(
        \Big(
            (
                p_2 \vee p_4
            ) \wedge (
                p_2 \vee \neg p_2
            )
        \Big) \wedge \Big(
            (
                p_4 \vee \neg p_4
            ) \wedge (
                \neg p_2 \vee \neg p_4
            )
        \Big)
    \bigg)
\end{gather*}
Finally, we can remove redundant parentheses and clauses, such as $p \vee \neg p$:
\begin{gather*}
    \Big(
        (
            \neg p_1 \vee p_2 \vee \neg p_3
        ) \wedge (
            p_1 \vee \neg p_2 
        ) \wedge (
            p_1 \vee p_3
        ) 
    \Big) \wedge (
        \neg p_5 \vee \neg p_1 \vee \neg p_4
    ) \wedge \Big(
        (
            p_2 \vee p_4
        ) \wedge (
            \neg p_2 \vee \neg p_4
        )
    \Big)
    \\
    \equiv
    \\
    (
        \neg p_1 \vee p_2 \vee \neg p_3
    ) \wedge (
        p_1 \vee \neg p_2 
    ) \wedge (
        p_1 \vee p_3
    ) \wedge (
        \neg p_5 \vee \neg p_1 \vee \neg p_4
    ) \wedge (
        p_2 \vee p_4
    ) \wedge (
        \neg p_2 \vee \neg p_4
    )
\end{gather*}
Thus, we have put \autoref{eq:CNF_example1} into CNF.
\subsection{Figures}
\begin{figure}[H]
    \centering
    \includegraphics[width=0.8\textwidth]{figures/Danske_Regioner__Sundhedsvæsenets_digitaliseringsrejse.png}
    \caption{The Danish healthcare sector's road to digitalisation}
    \medskip
    \small
    \raggedright
    The Danish healthcare sector's road to digitalisation, as presented by "Danske Regioner"\cite{Den-Reg-digitalisation}.
    \label{fig:healthcare_digitalisation}
\end{figure}

\begin{figure}[H]
    \centering
    \includegraphics[width=0.75\textwidth]{figures/Questionnaire_for_healthcare_sector.png}
    \caption{Questionnaire for healthcare personnel}
    \medskip
    \small
    \raggedright
    Questionnaire sent to Danish hospitals, in the pursuit of reaching out to healthcare personnel.
    \\
    \textbf{Translated from Danish:}
    \\
    \textbf{Title}: "Engineering project in the healthcare sector"
    \\
    \textbf{Introduction}: "Hello, my name is Felix, I am studying to become an engineer with a focus on Artificial intelligence at DTU. At the moment I am looking for a concrete problem for my master's thesis, and I was hoping to be able to work on a project, which combines technology with healthcare, in an attempt to make a real difference. In that pursuit, I wanted to ask whether you can spare a momemnt to answer this questionnaire." 
    \\
    \textbf{Question 1}: "What is the biggest and most frequent problem in your daily life as a healthcare professional?"
    \\
    \textbf{Question 2}: "What do you think is the biggest and most frequent problem in the daily life of your patients?"
    \\
    \textbf{Question 3}: "Please write your email and name, if I may contact you with follow-up questions."
    \\
    \textbf{Question 4}: "Name (not mandatory)"
    \\
    \textbf{Question 5}: "Email (not mandatory)"
    \label{fig:questionnaire_healthcare_personnel}
\end{figure}