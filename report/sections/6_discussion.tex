\section{Discussion}
\subsubsection*{The challenges presented by the Digital Taskboard}
We started this project by agreeing to provide OUH with a digital replacement for their current, manually transcribed taskboard (as detailed in \autoref{sec:Digital-Taskboard}). While true that this problem, on its own, couldn't provide a sufficient academic foundation for a Master's thesis in AI, we were nevertheless surprised by the scope of the problem and ultimately underestimated just how cumbersome\footnote{Which, to her great credit, our supervisor had actually warned us about.} of a task digitalising the taskboard could be.
\\
\\
The greatest obstacle for this part of the project was undoubtedly our inability to fulfil the role of IT consultant in conjunction with the lack of technical insight at OUH.
\\
More precisely, the difficulty was twofold. As we have no real experience in the role of IT consultant interacting with an actual 'client' (In our case, OUH) and as OUH isn't the most technically adept institution, we failed from the very onset to pinpoint the critical issues, which would later become complete technical bottlenecks. Here, we, in particular, recall the desire from OUH for the digital taskboard to have online updating capabilities. At one of the very first meetings we had with multiple staff representatives\footnote{It consisted of us, someone from IT, the head nurse of the department and our contact, a project and development nurse.}, we all agreed that this feature was necessary if the digital taskboard should replace the current physical whiteboard. If any one party had noticed the discrepancy in formatting between the spreadsheet where the names are updated and the spreadsheet where the names are initially extracted, we could have quickly determined whether the solution was even feasible or discussed how we should address the issue.
\\
Although this case best exemplifies our inability to assume the role of IT consultant, we nevertheless got to experience the eternal headache of consultancy, namely, the client's inability to define what they need. As illustrated in \autoref{fig:infographic}, the vast majority of our workflow was a deeply iterative process. One half consisted of working on some technical issue and then either realising there was some kind of inconsistency or, what we would consider a, bug in their data and then presenting it for OUH. For the other half, we implemented some feature which they had informed us would certainly wrap up the project, just to be told, when presenting the finalised version, that this other additional feature would, in reality, also be quite nice to have.
\\
Despite our many troubles developing software for real, authentic humans rather than in the neat, manageable setting presented at DTU, when it came down to the actual implementation of code, everything wasn't precisely "la vie en rose" either.
\\
\\
At a glance, digitalising the taskboard seemed no more complex than merging two 2D datasets, and we are still convinced it could have been were it not for OUH's currently in-place payroll and calendar overview system, HosInfo, serving as an ever-persistent thorn in our side. While the HosInfo implementation could be criticised plentily for its sluggish and featureless backend\footnote{Which we admittedly never had access to but only have seen in parsing when been guided through the system by the OUH staff.}, beyond the shadow of a doubt, our biggest gripe with it was its data handling or, rather, lack thereof. Not only does HosInfo make OUH's data nearly inaccessible by storing text data as pictures, but they also enforce no data formatting on their users. Additionally, rather than designing a solution capable of serving the greater planning effort at OUH, they have delivered a solution to a smaller subproblem while completely disregarding even the slightest notion of compatibility with the rest of OUH's IT environment.
\\
Such software makes it nigh impossible for OUH, not even to mention other external providers such as ourselves, to develop other solutions that need access to their data without forcing the end user to document everything many times over.
\\ 
However, HosInfo isn't the only third-party provider that is guilty of contributing to OUH's messy, tangled, and, at the same time, disconnected IT stack. We can only speculate on the reasons behind how it ends up as such. It might be due to the Danish and European procurement legislation~\cite{Udbudsloven, EU-Procurement-Legislation}, or in part due to a lack of technical insight at OUH, which limits them in seeing the greater scope of their administrative challenges. Nevertheless, it culminates in a system comprising a slew of individual IT solutions, which cannot share data nor benefit from what each provides.
\\
Trying to work around this constant impediment is perfectly summarised by this single response to our questionnaire, which loosely translated into English would be:
\begin{center}
 \textit{We have many systems that can do quite a lot, but not always in conjunction with each other, and most of the time, you have to work across several applications.}
\end{center}
While (our interpretation of) this statement perfectly encapsulates our immense frustration after six months of efforts of working with (or around) the Danish healthcare sector's IT, we can still scarcely imagine having to work with it on a daily basis. Therefore, we thoroughly believe that a more cohesive and all-encompassing platform would be the right move.
\\
We envision a cloud-based platform with data formatting requirements which can serve data to the correct users. Such a platform should have a single standardised user interface (UI) and allow the administrative workers to extract and log data for their corresponding hospitals (or other clinical facilities). It should facilitate an API allowing approved independent third-party providers to create individual software solutions that could work off of this platform. Such a platform would immediately eliminate any need to log data multiple times, as it would all be gathered in one place. Furthermore, if such a platform enforced a UI design guide onto third-party providers and required them to implement their solution onto the platform, it would save the healthcare personnel from having multiple credentials and remembering different layouts for each and every single individual application.
\\
This idea would require a lot from the IT sectors of the Danish healthcare sector; in turn, we also hope that it would be easier to maintain if all of it was gathered onto a single platform, with the single uniform goal of facilitating more seamless implementation of IT solutions.
\\
However, this isn't exactly 'new tech', as "Danske Regioner" in 2024 proposed a 'Danish Healthcare Cloud'~\cite{Den-Reg-digitalisation}. The referenced note touches upon many of the same topics as we have just detailed. Rather than feeling amiss for only arriving at the same conclusion as somebody else had found way before us, we will instead let it enforce our belief that such a restructuring wouldn't simply be an improvement; it is a requirement.
\\
\\
This brings us nicely along to our next point, for although, in all likelihood, we wouldn't be qualified to implement such a platform, we did attempt to build The Scheduling Assistant tool, which we believe could have been a good starting point for OUH, at least.
\\
As mentioned, digitalising the OUH taskboard didn't seem academically sufficient; therefore, we also pursued the scheduling task. We believed that we could implement this side-by-side with the digital taskboard. While we still think it technically feasible, as it has been done before~\cite{you-but-better} (and to a way greater extent), much like for the digital taskboard, we ran into multiple unforeseen difficulties.

\subsubsection*{The challenges presented by the Scheduling Task}
Whereas for the digital taskboard, the complications were primarily technical in nature, for the scheduling task, the bottleneck was very much the human factor.
\\
Firstly, most of the data required for planning a correct schedule isn't documented anywhere at OUH; it is simply some information that the administrative workers have learned through their practice. For us, acting as an external provider, this immediately complicates everything. If we are to develop a system or solution that could alleviate some of their more tedious tasks, that information is critical. However, for them to provide us with that, they must first halt their daily jobs and instead spend their time documenting this information. This is neither in the interest of the administrative workers, as we are only piling on to their already stacked workload, nor is it in the interest of their bosses because we don't provide any immediate value. Thus, we ended up working solely on a very small subproblem of the greater planning effort. This situation is obviously not ideal from the perspective of an engineer, as one would prefer to address the most challenging issues first to determine whether the modelling is feasible. From there, we could then scale down to smaller problems.
\\
\\
Yet, a huge topic for discussion concerning the scheduling task, regardless of the modelling complexity, is the interface design. As we touched upon at the end of \autoref{sec:scheduling-assistant}, the interface is in no way straightforward to design. Still, it is crucial to get it right in order for a solution to the scheduling task to be viable. 
\\
As we mentioned, the key is to design an interface which provides a seamless transition from informal constraints to coding logic. It is clear to us that some kind of abstraction is required; we are just uncertain as to exactly what shape it would take. However, we believe this abstraction must be found in collaboration with the end user. It should be designed based on iterative testing of a prototype. From there, it can be derived what constraints are easily defined as shown with \autoref{tab:Agent-Chart}, \autoref{tab:Agents}, \autoref{tab:Rolling-Chart} and \autoref{tab:Tasks} and whether more convoluted translations, could be taught to the end user or if that would ultimately mean the interface had failed to do its job. Here, we, in particular, consider the task which had the same name for every day of the week, but on a single day of the week, it was slightly more complex, requiring two agents instead of one. We ended up modelling this as two distinct tasks, but maybe the final interface should have some kind of toggle where the end user could specify just how many agents that specific task requires. This is just one example, and we believe that for a bigger problem, this could have been not only an intricate subject for research but also a very interesting one.

\subsubsection*{A last dicth effort: The Schedule Validator}
Regrettably, we failed to realise till it was probably too late that we hadn't accumulated enough goodwill with the administrative staff, which could provide us with the critical data. 
\\
While we had looked at the issue presented by the greater planning effort at OUH through a goal- and solution-oriented lens, we missed the smaller pieces that make up the larger picture.
\\
The planning and administrative staff had initially complained to us that they spent a significant amount of time simply checking whether their final schedule adhered to management's plan. We were far too quick to dismiss this problem, as we saw it would be made redundant by a scheduling assistant tool, which could simply plan a correct schedule the first time around.
\\
However, seen through the wise glasses of hindsight, this was an incredibly novice approach for multiple reasons. Firstly, it is probably always good advice to at least acknowledge the concerns of your stakeholders. Secondly, we simply failed to consider a test-driven implementation. We were so blinded by focusing on implementing a schedule planner that could adhere to their constraints and plan a correct schedule that we didn't consider there could exist methods other than CP to confirm the validity of the final schedule. This check would be a valuable, if not necessary, addition to any solution which is designed to serve the full planning procedure. Thirdly, it didn't help us gain the trust of the administrative staff, who were, and still are, the only ones who could provide us with the information necessary to create a functioning scheduling assistant tool. 
\\
As detailed in \autoref{sec:schedule-validator}, we eventually saw the light and started implementing this validator tool. Though, if given the time, we would still like to finalise this project, even if only to contribute something to OUH, we cannot help but question whether it is a good idea. To be frank, we are afraid of being another in a long line of software implementations that had insisted they keep on their floaties - rather than ever letting them try to swim on their own.
\\
More specifically, after having lambasted the current state of the Danish healthcare sector's IT plentily, we see the hypocrisy in simply adding another piece of disjointed software, which they would have learn to use and which does not function in tandem with the rest of OUH's tech stack.

\subsubsection*{Scalability}
We are still convinced that the approach we have laid out would work just as well in other clinical departments and, quite possibly, in smaller clinical institutions. From our conversations with OUH staff, even though multiple digital tools intended for shift registration exist, few, if any, ever truly automate the scheduling process; in practice, almost everyone produces the final schedules by hand. This arrangement may be tolerable in larger hospitals, where dedicated administrative personnel can handle the greater planning effort. However, from what we have heard (admittedly second-hand), some smaller clinics lack the budget or manpower for specialised planners, leaving senior doctors to assemble these schedules themselves. In such settings, a scheduling solution that can handle most of the tedious work would, arguably, be even more beneficial. Obviously, any successful rollout still depends on stakeholder acceptance and some willingness to integrate new technology into existing routines. Still, there is next to nothing in our design that prevents it from being applied to other environments once the local constraints have been formalised.

\subsubsection*{GDPR considerations}
A final yet crucial consideration is the security and privacy of the data being processed. Any solution handling personnel schedules or patient-related documentation must comply with appropriate regulations, such as GDPR~\cite{EU-GDPR}, which requires clear definitions of data ownership and responsible data controllers. While we certainly do not claim the expertise to deliver a fully GDPR-compliant system, we still acknowledge that any production-ready tool must incorporate both encryption standards and access controls in order to guarantee the confidentiality of the stored data. However, this would, in turn, add further to OUH's already stacked workload, as they would be required to define protocols for system access, user actions, logging and more. If such regulations are ignored and the developers fail to implement the necessary safeguards, regardless of how seamlessly designed or efficiently optimised the system is, it ultimately risks falling short due to legal and ethical complications - potentially resulting in a final product that is never adopted.
\\
\\
All things considered, no matter the software implementation, a rigorous mixture of technical skills, stakeholder and project management, and thoughtfully designed UI is necessary for it to be adapted successfully into the Danish healthcare sector. In turn, if the sector actually desires modernisation through the automation of their more tedious and less clinical tasks, they must be prepared to take the necessary steps for GDPR compliance, data formatting and procedural restructuring. 