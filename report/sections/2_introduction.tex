\section{Introduction}
The Danish healthcare sector is characterized by highly skilled medical professionals operating under constant and evergrowing time pressure.
Denmark's population is increasingly aging, with a trend toward an older demographic. Statistics Denmark\cite{dst-older-pop} states that the number of Danish citizens aged $80$ and above is expected to increase by $43$\% within the next decade, while the group aged $60$-$79$ is also expected to grow by $6.8$\%\footnote{The numerical increase is from $319,874$, in $2024$, to $457,488$, in $2034$, for $80$+ and $1,279,671$, in $2024$, to $1,359,430$, in $2034$ for $60$-$79$}. Moreover, recruiting qualified healthcare personnel is becoming increasingly difficult. The Ministry of Higher Education and Science\cite{MoHEaS-healthcare-rectruitment} reports that, although there has been a recent increase in the number of students graduating in healthcare-related fields, this is still insufficient and the lack of qualified healthcare personnel is only expected to grow.   
