\section{Introduction}
The Danish healthcare sector is characterised by highly skilled medical professionals operating under constant and evergrowing time pressure.
Denmark's population is increasingly ageing, with a trend toward an older demographic. Statistics Denmark\cite{dst-older-pop} states that the number of Danish citizens aged $80$ and above is expected to increase by $43$\% within the next decade, while the group aged $60$-$79$ is also expected to grow by $6.8$\%\footnote{The numerical increase is from $319,874$, in $2024$, to $457,488$, in $2034$, for $80$+ and $1,279,671$, in $2024$, to $1,359,430$, in $2034$ for $60$-$79$}. Moreover, recruiting qualified healthcare personnel is becoming increasingly difficult. The Ministry of Higher Education and Science\cite{MoHEaS-healthcare-rectruitment} reports that, although there has been a recent increase in the number of students graduating in healthcare-related fields, this is still insufficient, and the lack of qualified healthcare personnel is only expected to worsen over time. The Ministry of Finance\cite{MoHEaS-healthcare-rectruitment} anticipates that the healthcare sector will be short by approximately $15,000$ SOSU workers by $2035$, and in response, the government plans to partner with developing countries\footnote{Mentioned are India and the Phillipines.} to help recruit healthcare personnel.
\\
\\
Over the past three decades\footnote{See \autoref{fig:healthcare_digitalisation}, 'Danske Regioner' made a pleasant visualisation.}, digital initiatives and platforms have been developed in an attempt to alleviate the challenges posed by an understaffed healthcare sector by enhancing workflow efficiency and patient care. However, due to the complexity within the healthcare sector arising from diverse patient needs, intricate protocols and procedures, and a plethora of administrative tasks, the resulting systems put in place\footnote{The classic example here being "Sundhedsplatformen", Epic's healthcare-IT system, that since its inception in 2016, has been under constant critique\cite{DR-healthcare-platform, Altinget-healthcare-platform, TV2-healthcare-platform}. Though, in all honesty, in 2021 in a user survey by "Center for Patientinddragelse"\cite{SP-user-survey} they noticed a slight uptick in general satisfaction amongs users i.e. the healthcare staff.} are often challenging for healthcare personnel to use effectively. While the initiative behind these solutions is commendable, the final products usually falls short. The solutions can be cumbersome, unintuitive, and prone to disrupt daily routines. In this context, the importance of human-centred design and usability principles cannot be overstated. Without careful attention to user experience, even the most technologically advanced solution risks going unused, ultimately failing to support the healthcare personnel it was designed to assist.


