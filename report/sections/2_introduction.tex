\section{Introduction}
The Danish healthcare sector is characterized by highly skilled medical professionals operating under constant and evergrowing time pressure and stringent regulatory requirements. Over the past decade, a wave of digital initiatives and platforms has been introduced with the aim of improving workflow efficiency and patient care. However, the complexity of the healthcare environment—intersecting patient needs, rigorous documentation standards, and numerous administrative tasks—often results in systems that are challenging for staff to use effectively. While the intent behind these solutions is laudable, the reality often falls short: digital platforms can be cumbersome, unintuitive, and prone to disruptions in daily routines. In this context, the importance of human-centered design and usability principles cannot be overstated. Without careful attention to user experience, even the most technologically advanced solution risks going unused, ultimately failing to support the healthcare professionals it is meant to assist.

This thesis emerges from a desire to address tangible, everyday digital challenges faced by hospital staff—professionals who are highly trained in patient care but not necessarily equipped to navigate complex IT systems. Initially, the vision was to leverage cutting-edge techniques such as Large Language Models (LLMs) to streamline documentation processes. However, direct engagement with the sector revealed a set of more immediate and pressing needs. Instead of pursuing an advanced, theoretically appealing AI solution, the focus pivoted toward more foundational digital tools. These were twofold: first, automating a taskboard system, which currently requires manual data transcription by a nurse at the start of each day; and second, developing a scheduling assistance tool to help assign tasks to staff efficiently and fairly, considering a range of practical constraints.

Neither of these tasks was as straightforward as anticipated. While the initial concept for the taskboard automation suggested a relatively simple data-integration challenge, the implementation process has extended over several months due to unforeseen complexities in data formats, infrastructure limitations, and evolving requirements. Similarly, the scheduling tool—conceived as a straightforward application of constraint programming or operations research methods—became mired in complications when critical data and constraints were difficult to elicit from stakeholders. This experience highlights the atypical and deeply iterative nature of the project. Unlike traditional academic assignments, in which problems and data are neatly defined, this endeavor required persistent negotiation, creative problem-solving, and an adaptable mindset. Requirements changed mid-development; data were elusive or incomplete; and design specifications, when met, often had to be revisited because they no longer served the evolving understanding of the hospital's needs.

In light of these challenges, the primary objectives of this thesis are threefold:

Understanding the Digital Challenges: Investigate the nature of the digital and organizational issues within a hospital department through a combination of questionnaires, direct communication, and iterative development cycles.
Automating the Taskboard: Develop a solution to integrate and visualize staff schedules and tasks, reducing the manual effort required each morning and ensuring that the representation is both accurate and accessible to non-technical users.
Scheduling Assistance Tool: Prototype an intelligent scheduling agent using operations research techniques, exploring how constraints and heuristics can be balanced to produce equitable and efficient staff rosters.
Beyond these technical aims, this thesis seeks to reflect on the process itself—how the theory and skills gained in a Master's program translate (or fail to translate) into real-world practice. The journey documented here stands as a valuable lesson in the complexities of stakeholder management, the pitfalls of underspecified requirements, and the necessity of iterative refinement. By sharing not only the final solutions but also the struggles and missteps along the way, this work underscores the value of human-centered methods. It illustrates that meaningful progress often hinges less on deploying cutting-edge algorithms and more on understanding the nuanced realities of end-users who will rely on these systems every day.