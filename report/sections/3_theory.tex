\section{Theory}
Of the three main tasks explored in this thesis, developing a \textit{scheduling assistance tool}, which we henceforth will refer to as '\textit{the scheduling task}', required the most mathematical approach. \textit{The scheduling task} is a case of the \textbf{Generalised Assignment Problem}\cite{Wiki-general-assignment-prob}, which can ultimately be expressed as a \textbf{Linear Program}\cite{Wiki-linear-programming}, specifically, a \textbf{Binary Integer Program}.
The following sections explore different theoretical approaches that might yield a satisfactory solution to such a \textbf{Generalised Assignment Problem}.

\subsection{Genetic Algorithms (GAs)}
A GA, as first introduced by John Holland\cite{Genetic-Algorithm-original} in 1975\footnote{Though later revised in 1992.}, is a metaheuristic within primarily \textbf{Computer Science} and \textbf{Operations Research} (OR). Inspired by evolutionary theory's natural selection, a GA is typically employed for \textbf{Optimisation} and \textbf{Search problems}.