\section{Conclusion}
For this project, we set out to address the daily real-world challenges clinicians in the Danish healthcare sector face. We sought to accomplish this through three different projects: digitalising an essential taskboard at \acrshort{ouh}, developing a scheduling assistant tool, and an automatic schedule validator. While we managed to produce prototypes for merging differing datasets and modelling a scheduling task, we continually faced technical, bureaucratic, and human-centred challenges, which all serve to emphasise how real progress depends on more than solid code alone. Our experience suggests that, for any new digital solution to succeed in the Danish healthcare sector, key factors, such as GDPR compliance, IT integration, and end-user satisfaction, must be thoroughly considered from the very onset. Nevertheless, we remain convinced that automating these daily administrative tasks would not only be feasible but highly beneficial if the sector was willing to commit to clarifying constraints, standardising and ensuring the formatting of data, not to mention allowing enough time for new technology to be integrated with existing routines.