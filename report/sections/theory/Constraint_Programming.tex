\subsection{Constraint Programming}
Another approach for solving the Generalised Assignment Problem, which the \textit{scheduling task} poses, is Constraint Programming (CP)\footnote{Constraint Programming is a somewhat odd nomenclature, as \textit{programming} here refers to 'the process of scheduling' rather than in the sense of a computer programming language.}. CP is concerned with problems that require a feasible solution. Often, this involves searching for a "needle within a haystack", as CP aims to arrive at a solution satisfying a complex set of constraints. As such, CP isn't inherently an optimisation technique and won't necessarily make use of an objective function. 
\\
However, in order to arrive at CP, we will first argue, without a complete theoretical deep dive, that the Generalised Assignment Problem can be reformulated into the binary case of an integer linear problem (ILP). A binary ILP takes the form\cite{Integer-Programming-Book} 
\begin{align*}
    \text{max} \quad &\sum_{i = 1}^m \sum_{j = 1}^n c_{ij} \cdot x_{ij}
    \\
    \text{subject to} \quad &\sum_{j = 1}^n a_{ij} \cdot x_{ij} \leq b \qquad &\text{for} \quad i = 1, \hdots, m
    \\
    &\sum_{i = 1}^m x_{ij} \leq 1 \qquad &\text{for} \quad j = 1, \hdots, n
    \\
    \text{for} \quad &x \in \{0, 1\}^{m \times n},
\end{align*}
where $x$ is the binary decision variable. This formulation gives way to a Boolean Satisfiability Problem (BSAT), a subfield of CP, which we can attempt to solve with a \textit{satisfiability} (SAT) solver. In the following sections, we will elaborate on different algorithms employed by SAT solvers.



