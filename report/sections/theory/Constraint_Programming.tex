\subsection{Constraint Programming}
Another approach for solving the \textit{scheduling task} is Constraint Programming (CP)\footnote{Constraint Programming is a somewhat odd nomenclature, as \textit{programming} here refers to 'the process of scheduling' rather than in the sense of a computer programming language.}. CP is concerned with problems that require a feasible solution. Often, this involves searching for a "needle within a haystack", as CP aims to arrive at a solution satisfying a complex set of constraints. As such, contrary to the GA scheme, CP isn't inherently an optimisation technique and won't even necessarily use an objective function.
\\
As we will argue later, the domain posed by the \textit{scheduling task} gives way to a subfield of CP called satisfiability (SAT). Satisfiability concerns checking whether a boolean formula holds (is satisfiable) under some truth assignment to its variables. SAT solvers are explicitly made for solving such problems, so in the following sections, we will elaborate on different algorithms employed by such solvers.

\subsubsection{Conjunctive Normal Form}
However, although not inherent to boolean satisfiability, most SAT solvers work on boolean formulas in Conjunctive Normal Form (CNF). Therefore, we will first introduce CNF.
\\
From Ben-Ari (chapter 4)\cite{Math-Logic-for-CompSci} we have the following definition:
\begin{definition}
    A formula is in conjunctive normal form (CNF) if and only if it is a conjunction of disjunctions of literals.
\end{definition}
Here, conjunction and disjunction refer to the logical connectives $\wedge$ and $\vee$, respectively. While a formula and a literal, as given by Ben-Ari (chapter 2)\cite{Math-Logic-for-CompSci}, can be formally defined as:
\begin{definition}
    A formula in propositional logic is a tree defined recursively:
    \begin{itemize}
        \item A formula is a leaf labeled by an atomic proposition (often shortened
        to atoms).
        \item A formula is a node labeled by $\neg$ with a single child that is a formula.
        \item A formula is a node labeled by one of the boolean operators with two children both of which are formulas.
    \end{itemize}
\end{definition}
\begin{definition}
    A literal is an atom or the negation of an atom. An atom is a positive
    literal and the negation of an atom is a negative literal. For any atom $p$, $\{p,\neg p\}$ is a complementary pair of literals.
    \\
    For any formula $A$, $\{A,\neg A\}$ is a complementary pair of formulas. $A$ is the complement of $\neg A$ and $\neg A$ is the complement of $A$.
\end{definition}
We will now briefly discuss how to arrive at CNF.

\paragraph{Propositional logic to Conjunctive Normal Form}
Ben-Ari (chapter 4) gives us the following theorem:
\begin{theorem}
    Every formula in propositional logic can be transformed into an equivalent formula in CNF.
\end{theorem}
Which is accompanied by a 



\subsubsection{DPLL}



